\section{Social Engineering}\label{sec:social_engineering}
Dieses Kapitel beschreibt den Begriff Social Engineering und grenzt die dazu gehörenden Techniken von
üblichen Vorgehensweisen des Hackens ab. Dabei wird in \Kapitel{sec:definition} zunächst ein
Definition gegeben.
In \Kapitel{sec:alternative_zu_technischen_methoden} wird die Einfachheit solche Angriffe gegenüber
technischer Angriffe hervorgehoben.
Zum Schluss werden in \Kapitel{sec:gangige_angriffe} Beispiele für verschiedene Methoden gegeben.

\subsection{Definition}\label{sec:definition}
Um den Begriff Social Engineering korrekt einordnen zu können müssen zunächst herkömmliche Aspekte der
Informationssicherheit betrachtet werden.
Bei diesen handelt es sich zum einen um physikalische Zugriffskontrolle (z.B. Identitätsprüfungen an
Türen)und zum anderen um IT-Sicherheit (meistens wird allerdings bei IT-Sicherheit lediglich von
„Sicherheit“ gesprochen).
Diese beiden Sicherheitsaspekte haben zweifellos ihren Platz und ihre Berechtigung.
In vielen Fällen werden aber nur diese Art Angriffe berücksichtigt, denen solche Systeme
entgegenarbeiten sollen.
Offensichtlich ist, dass sich IT-Sicherheit und physikalische Sicherheit ausschließlich auf ein
Unternehmen beschränken, welches lediglich aus IT-Systemen und Gebäuden mit Türen und Fenstern
bestehen.
Eine der essenziellsten Kernkomponenten des Unternehmens wird dabei gänzlich übersehen.
Der Mitarbeiter stellt das größte Kapital des Unternehmens dar. Durch Social Engineering Techniken
wird er jedoch zugleich zur größten Sicherheitslücke eines Unternehmens.
Dies liegt größtenteils an den mangelnden Gegenmaßnahmen die es zu Social Engineering Attacken gibt. \cite{hacking-the-human}

Nun stellt sich weiter die Frage, was einen solchen Angriff ausmacht.
Social Engineering Attacken zielen darauf ab, bestimmte Personen dahingehend zu manipulieren bestimmte
Informationen herauszugeben oder Handlungen auszuführen, für die der EIngreifer selbst keine
Berechtigung besitzt.
Um solche Handlung auszulösen werden verschiedenste Techiken zur Täuschung herangezogen, die alle auf
psychologischen Erkenntnissen beruhen.
Auf diese wird in \KapitelAndName{sec:psychologische_grundlagen} tiefer eingegangen.

\subsection{Alternative zu technischen Methoden}\label{sec:alternative_zu_technischen_methoden}
In den letzten Jahren erfreuten sich Social Engineering Angriffe zunehmender Beliebtheit.
Diese Entwicklung ist nicht ohne Grund zu beobachten.
Während seit Beginn des Informationszeitalters auch die Angriffe auf Datenbestände immer häufiger und
vor allem gefährlicher geworden sind, wurden entsprechend starke Gegenmaßnahmen entwickelt.
Diese beschränken sich bis heute auf die Apsekte der physikalischen Sicherheit und der IT-Sicherheit.
Auch heute noch spielt vor allem die IT-Sicherheit in vielen Unternehmen sicherlich gerechtfertigt
eine übergeordnete Rolle.
In jedem Unternehmen finden sehr wirkungsvolle jedoch ebenso kostenintensive Abwehrmechanismen ihre Anwendung.
Diese stellen zwar kein unüberwindbares Hindernis dar, halten aber dennoch vielen Angriffen stand oder
schrecken bereits vor einem Versuch ab.
Während große Sicherheitsfirmen hierfür teure Software- und Hardwarelösungen anbieten, gibt es derzeit
kein besonders großes Angebot an Abwehrmaßnahmen, die der Prävention von manipulativen Angriffen auf
die Mitarbeiter dienen. \cite{hacking-the-human}

Es liegt auf der Hand, dass immer mehr Angriffe nicht herkömmlich auf die IT-Systeme direkt
gerichtet werden, sondern entsprechende Mitarbeiter als Ziel haben.
Der zu konventionellen Methoden gesparte Aufwand ist größer als er erwartet wird.
Im nächsten Kapitel werden einige gängige Methoden und deren Effizienz vorgestellt.

\subsection{Typische Angriffsarten}\label{sec:gangige_angriffe}
In diesem Kapitel werden exemplarisch drei grundlegende Angriffstechniken vorgestellt und es wird grob geklärt, warum diese Techniken besonders große Erfolgschancen bieten.
Dabei werden einige psychologische Grundlagen vorweggenommen, welche in \Kapitel{sec:psychologische_grundlagen} detailliert
beschrieben werden.
Zum Verständnis der folgenden Beispiele sind diese Grundlagen nicht notwendig, jedoch werden nach der
Lektüre des Kapitels \nameref{sec:psychologische_grundlagen} die Vorgänge hinter diesen Beispielen noch sehr viel deutlicher erkennbar sein.
Bei den im folgenden vorgestellten Angriffskategorien handelt es sich um Phishing, Phone Elicitation und Identitätsbetrug. Diese Herangehensweisen haben viele Gemeinsamkeiten, unterscheiden sich jedoch in der Anonymität des Angreifers.

\subsubsection{Phishing}
Beim Phishing handelt es sich um die unpersönlichste Variante eines Social Engineering Angriffes.
Ziel einer Phishing-Attacke ist es mittels einer E-Mail, das Angriffsziel dazu zu bringen einen infizierten Anhang zu öffnen, oder einen bestimmten Link zu besuchen. Man unterscheidet hierbei zwischen dem gewöhnlichen Phishing (das Senden von Massenmails) und dem Spear-Phishing, bei dem die Mail(s) zielgerichtet versendet werden. Handelt es sich beim Opfer einer Spear-Phishing-Attacke um eine besonders einflussreiche Person, bezeichnet man dies als \fachwort{Whaling}. \cite{hadnagy}

Das Phishing bietet durch den ausbleibenden persönlichen Kontakt mit der Zielperson ein besonders hohes Maß an Anonymität. Somit stellt ein Angriff dieser Art meist kein großes Risiko für den Social Engineer dar.
Allerdings erschwert die Anonymität es dem Angreifer das Vertrauen der Zielperson zu erwecken.
\cite{hacking-the-human}

Phishing-Mails zielen immer auf bestimmte Gefühle der Opfer ab. Dazu gehört bspw. Angst vor Diebstal oder Verlust, wenn in der Mail beschrieben wird, dass der eigene PC von Viren befallen ist oder man eine Mahnung für eine Rechnung erhält.
Der Angreifer kann ebenso auf andere Emotionen wie Trauer (Hilfe-Aufrufe) oder Freude (\speech{Sie haben gewonnen!}) abzielen.
Der Angreifer versucht durch das auslösen bestimmter Emotionen ein bestimmtes Verhalten auszulösen. Dabei handelt es sich um Automatismen, die in jedem Menschen ablaufen können. In \Kapitel{sec:automatismen} wird dieses Phänomen noch genauer beschrieben werden. \cite{hadnagy}


\subsubsection{Phone Elicitation}\label{sec:phone-elicitation}
Das Abhören per Telefon stellt für den Angreifer einen guten Kompromiss zwischen Anonymität und Vertrauen der Zielperson dar. Das Opfer kann zwar die Identität des Anrufers nicht vollständig nachprüfen, es gibt allerdings trotzdem einige Indizien, die es möglich machen die Zielperson von der Echtheit des Anrufers zu überzeugen. So erweckt es bspw. schnell Vertrauen, wenn man die angezeigte Telefonnummer dem eigenen Unternehmen oder einem Partner-Unternehmen zuordnen kann. Wenn die Zielperson zudem beiläufig eine Bemerkung macht, aus der man schließen kann, dass er zum Unternehmen gehört (Erwähnen eines anderen Mitarbeiters, o.ä.).

Diese Indizien sind in der Praxis sehr leicht zu fälschen. Mittels \fachwort{Spoofing} ist es mit besonders geringem Aufwand möglich die auf dem Display angezeigte Rufnummer zu verfälschen. Anhaltspunkte für Randinformationen wie die Existenz eines bestimmten Mitarbeiters lassen sich durch vorbereitende Analyse ebenfalls mit überschaubarem Aufwand ermitteln.
\cite{hadnagy}

Beliebte Angriffsziele für Angriffe per Telefon sind vor allem Heimarbeiter und Mitarbeiter des IT-Helpdesks. Im Folgenden wird dieses Angriffsziel näher beleuchtet.


\subsubsection*{Heimarbeit und Helpdesks}\label{sec:heimarbeitundhelpdesks}
Ein beliebtes Ziel für Social Engineering Angriffe stellen Mitarbeiter dar, die vom Home-Office aus
arbeiten.
Dabei können die Mitarbeiter direkt als Ziel genommen werden. Mitarbeiter, die zu großen Teilen von zu
Hause aus arbeiten, wissen oftmals nicht über alle Kollegen Bescheid und können auf Anrufe eines
Angreifers, der sich als vermeintlicher Arbeitskollege ausgibt, vertrauliche Informationen
herausgeben. Dabei wird Distanz zum Unternehmen als Schwachstelle ausgenutzt.

Dieses Angriffsziel ist zudem in die andere Richtung denkbar.
Dabei wird der IT-Helpdesk als Zielscheibe gewählt.
Mitarbeiter eines solchen Helpdesks sind oft geschult darauf freundlich und zuvorkommend zu handeln.
Da gegenüber Heimarbeitern eine besonders große Hilfsbereitschaft an den Tag gelegt wird, ist es eine
beliebte Taktik sich als Heimarbeiter auszugeben und so gewünschte Informationen zu erhalten. \cite{hacking-the-human}

Bei Angriffen auf Mitarbeiter des Helpdesks oder Heimarbeiter macht sich der Social Engineer die Rollen zu Nutze, die diese Personen im Arbeitsalltag spielen. Dazu gehört auch die Autorität durch Wissen des Helpdesk-Mitarbeiters, die für den Angriff auf einen Mitarbeiter im Home-Office ausgenutzt werden kann (genauer beschrieben in \KapitelAndName{sec:autorität-und-befehle}), sowie die grundsätzliche Hilfsbereitschaft, die für den umgekehrten Angriff genutzt werden können.


\subsubsection{Identitätsbetrug}\label{sec:identitätsbetrug}
Beim Identitätsbetrug geht die Anonymität des Angreifers gegen null, da er der Zielperson direkt gegenübersteht. Wird der Angriff jedoch gekonnt durchgeführt, ist es die wirkungsvollste Art eines Social Engineering Angriffes, da mit dieser Methode sehr schnell Vertrauen aufgebaut werden kann.
Ein passender Begriff hierfür wäre \fachwort{Real Life Acting} oder \fachwort{Angewandtes Schauspiel}.
Mit verschiedenen Hilfsmitteln wie Kleidung, Sprache, Körperhaltung, gefälschten Identitätskarten und einem kongruenten Pretext überzeugt der Social Engineer seine Zielpersonen meist bereits ohne Worte von der Echtheit seines Auftretens. \cite{hadnagy}

Die Dauer eines solchen Angriffes kann dabei stark variieren. Meist handelt es sich um eine kurze Infiltration, bei der Positionen von Überwachungskameras bestimmt oder mit Malware verseuchte USB-Sticks  platziert werden. Wie solche Angriffe ablaufen können wird im folgenden Abschnitt kurz erläutert.
Eine andere Dimension dieser Kategorie ist das, was man im allgemeinen als verdeckten Ermittler, Spion oder \ugspr{Maulwurf} versteht. Dies stellt eine sehr aufwändige und ebenso gefährliche Variante des Social Engineering dar und wird später im Abschnitt \nameref{sec:neue-angestellte} anhand eines kurzen Beispiels erläutert. \cite{hadnagy}\cite{hacking-the-human}

\subsubsection*{Hilfsmittel zum Identitätsbetrug}\label{sec:hilfsmittel-zum-identitätsbetrug}
Die zur Verfügung stehenden Mittel sind überraschend simpel. Ziel dieser Methoden ist es allerdings immer unausgesprochene Fragen über die Person des Angreifers automatisch zu beantworten, so dass sich Zielpersonen keine weiteren Gedanken darüber machen, ob es sich bei dem Social Engineer um einen Betrüger handelt. Dabei haben sich vor allem drei charakteristische Muster bewährt.

Dabei handelt es sich zum einen um unzureichend geschützte Bereiche eines Gebäudekomplexes. Gemeint sind damit insbesondere Raucherbereiche. Über solche Areale erhält man meist auch ohne Besitz einer Zugangskarte physischen Zugriff auf das Gebäude des Zielunternehmens.
Die Herausforderung des Social Engineers liegt dabei in erster Linie darin, die Rolle des Mitarbeiters, der von einer Zigerettenpause zurückkehrt, überzeugen zu spielen.

Eine weitere Methode um schnell Zugriff zu einem gesicherten Gebäude zu erhalten oder sich unbemerkt fort zu bewegen ist es, schwere Gegenstände oder einen großen Karton zu tragen. Mitarbeiter hinterfragen meist nicht, was transportiert wird oder um wen es sich bei der Person handelt. Im Gegenteil wird dem Social Engineer meist sogar die Türe aufgehalten.

Die letzte nennenswerte Methode ist das Verwenden einer falschen Zugangskarte. Dafür genügt es die Karte echt aussehen zu lassen. Versucht der Angreifer nun mehrmals vergeblich mit seiner falschen Karte Zugang zu erhalten, kann er sich an den nächsten Mitarbeiter wenden und ihn darum beten, ihn einzulassen.
\cite{hadnagy}

Vor allem die beiden zum Schluss genannten Methoden machen sich dem im Allgemeinen als \ugspr{Helferinstinkt} bekannten Phänomen zu Nutze. Dies geht stark einher mit dem Bedürfnis gemocht zu werden einher. In \KapitelAndName{sec:psychologische_grundlagen} wird dieses Prinzip nochmals aufgegriffen und genauer erklärt.

% Bezug auf psychologisches Konzept: Menschen wollen gemocht werden!

\subsubsection*{Neue Angestellte}\label{sec:neue-angestellte}
Dieses Szenario ist zwar etwas seltener anzutreffen jedoch nicht zu unterschätzen, da auf diese
Weise bereits Zugriffsrechte auf bestimmte Bereiche des Unternehmens oder der Datenbasis gewährt wird.
Auch in solchen Situationen wird die Hilfsbereitschaft von anderen Mitarbeitern ausgenutzt, denn
gerade neue Mitarbeiter benötigen am Anfang viel Hilfe.
Außerdem bringt man neuen Mitarbeitern mehr Nachsehen entgegen.

In einer solchen Position ist es für einen Angreifer ein leichtes an eine Vielzahl wichtiger
Informationen zu kommen oder zusätzliche Schwachstellen ausfindig zu machen.

Wenn neue Mitarbeiter nach kurzer Zeit ohne plausiblen Grund Kündigen, kann dies ein Indiz dafür sein,
dass es sich um einen Angriff gehandelt haben kann.
Um solche Situationen zu verhindern, ist es angebracht für neue Mitarbeiter ausreichende Background-Checks vorzunehmen. \cite{hacking-the-human}

\subsection{Wirkungsweise}\label{sec:wirkungsweise}
% Zusammenfassung: Was haben diese Methoden gemein?
% Vertrauen aufbauen
% Glaubwürdigkeit!
% Brücke bauen zu Risikomanagement!