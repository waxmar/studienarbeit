\section{Umfrage}\label{sec:umfrage}
Bis hierhin ist geschildert worden, worum es sich bei Social Engineering handelt, welche Techniken dabei zum Einsatz kommen und wie das damit verbundene Risiko einzuschätzen ist.
Um die tatsächlichen Anfälligkeiten zu überprüfen ist im Rahmen der Studienarbeit eine Umfrage durchgeführt worden, bei welcher verschiedene berufstätige Personen zu verschiedenen Themen befragt wurden.
Die Umfrage mit dem Titel \ugspr{Hilfsbereitschaft und Social Engineering} und die daraus gewonnenen Ergebnisse werden im folgenden vorgestellt, aufbereitet und diskutiert.
Ebenfalls wird die Vorgehensweise skizziert.

\subsection{Fragebogen}
Für die Verteilung und Erstellung wurde das Web bzw. die Software Limesurvey verwendet.
Die Umfrageergebnisse sind mittels einiger demographischer Daten kategorisiert.
Bei den für diese Umfrage erhobenen Daten handelt es sich um das Alter, das Geschlecht, die Berufsgruppe, die Unternehmensgröße sowie die Anzahl der Standorte.
Im weiteren Verlauf werden Fragen zu verschiedenen Themenschwerpunkten gestellt.
Dabei handelt es sich konkret um Fragen zu Hilfsbereitschaft, Vertrauen, Aufmerksamkeit und einigen abschließenden Fragen zu Social Engineering selbst.

Im ersten Abschnitt (Hilfsbereitschaft) werden Fragen zu Situationen im Arbeitsalltag gestellt.
Diese Szenarien sind maßgeblich an den in Kapitel \ref{sec:gangige_angriffe} vorgestellten Angriffsmustern orientiert und sollen die Anfälligkeit der Probanden nachweisen.
Bei den Fragen handelt es sich um Ja/Nein-Fragen, bei denen angegeben werden soll, ob man dazu tendiert die erwartete Handlung auszuführen oder diese zu unterlassen.
Bei den ersten drei Fragen handelt es sich um direkte Angriffe bei denen eine gesamte Identität vorgetäuscht wird, die letzte Frage bezieht sich auf die Technik Phone Elicitation.
Zudem überprüft die letzte Frage die Mitteilungsbereitschaft gegenüber den anderen Geschlecht.

Die darauf folgende Serie von insgesamt sechs Fragen bezieht sich auf das Vertrauen der Probanden.
Dabei gibt es Fragen zu allen der drei Angriffskategorien: Identitätsbetrug, Phone Elicitation und E-Mail-Phishing.
Dabei wird das Vertrauen in die Identität von angezeigten Telefonnummern und E-Mail-Adressen (und somit indirekt das bewusste Wissen über Spoofing-Techniken) überprüft.
Damit ist die Problematik gemeint, dass eine Person tatsächlich über dieses Thema informiert ist, in der konkreten Situation jedoch nicht daran denkt.
Die letzten beiden Fragen zeigen Bilder von jeweils einer Person und sollen prüfen inwiefern die Optik das Vertrauen beeinflusst.
Innerhalb dieser Serie sollen Bewertungen des Vertrauens auf einer Skala von eins bis fünf angegeben werden.
Dabei steht eins für \ugspr{kein Vertrauen} und eine fünf für \ugspr{volles Vertrauen}.

In der folgenden Kategorie werden dem Probanden drei Bilder gezeigt, welche jeweils verschiedene Personen darstellen.
Zum einen wird eine Reinigungskraft gezeigt, welche gerade einen Fußboden wischt, zum anderen ein Techniker, der gerade an einer Art Verteilerkasten steht sowie eine Führungskraft.
Dabei soll der Proband angeben wie sehr er diese Personen im Arbeitsalltag wahrnimmt.
Wie in der vorangegangen Kategorie wird dies mittels einer Skala von eins bis fünf festgestellt.
Abschließend werden Fragen zu Social Engineering gestellt.
Der Proband wird dabei gefragt, ob er von Techniken wie Spoofing bereits gehört hat.
Außerdem soll er eine persönliche Einschätzung des Risikos abgeben.

Die Umfrage selbst ist für den Zeitraum von ca. sechs Wochen öffentlich verfügbar gewesen und ist von freiwilligen Probanden durchgeführt worden.
Verteilt wurde die Umfrage weitestgehend mittels sozialer Netze.

% Bezüge auf psychologische Grundlagen machen

\subsection{Nachträgliche Betrachtung}
Im Rückblick vor allem jedoch durch die Betrachtung der Ergebnisse sind einige Punkte verdeutlicht worden, welche dem Fragebogen Aussagekraft entziehen bzw. Potenzial für weitere Aussagen entzogen haben.
Innerhalb der demographischen Daten ist es wie den Ergebnissen des Fragebogens zu entnehmen ist, leider nicht gelungen, die Berufsgruppen sinnvoll abzudecken.
Zwar können Kernaussagen darüber getroffen werden, inwiefern sich IT-affine Berufsgruppen von solchen Unterscheiden, die sich nicht näher damit beschäftigen, jedoch ist innerhalb letzterer keine Möglichkeit gegeben genauere Erkenntnisse zu gewinnen.
Eine granulare Aufteilung ist an dieser Stelle sicherlich interessant und würde für verschiedene Bereiche in Unternehmen einen sinnvollen Aufschluss darüber geben, welche Abteilungen leichter anzugreifen sind als andere.
Auch sind andere Merkmale gar nicht erfasst worden wie z.B. die Dauer der Berufstätigkeit oder wie viele verschiedene Berufsgruppen an einem Standort vertreten sind. % Merkmale nennen

Die Fragen, welche Personen zeigen, haben seitens der Auswertung ein großes Potenzial, welches aufgrund der begrenzten Umfragedauer nicht vollständig ausgeschöpft werden konnte.
So könnte man zur Fragegruppe Wahrnehmung wesentlich detailliertere Erkenntnisse gewinnen.
Gerade in Kombination mit genaueren Angaben im Bereich der Berufsgruppe können hierbei sicherlich nützliche Hinweise herausgearbeitet werden.

Alldem ist hinzuzufügen, dass es sich bei den Ergebnissen letztlich nicht um eine Überprüfung des Ernstfalls handelt sondern lediglich eine theoretische Befragung vorliegt.
Unterschiede zu einer Situation wie sie tatsächlich vorkommt liegen vor allem in der persönlichen Distanz der Person.
Wird man auf einem Fragebogen mit einer solchen Situation konfrontiert ist es wesentlich besser möglich sich davon zu distanzieren und klarer darüber nachzudenken.
Zudem ist dem Probanden dadurch mehr Zeit gegeben.
Zwar sind die Probanden zum einen darauf hingewiesen worden, dass es sich um ein anonyme Umfrage handelt, zum anderen wurde darum gebeten die Fragen so zu beantworten, dass es dem eigenen Verhalten entspricht, jedoch ist nicht auszuschließen, dass ein kleiner Anteil der Anworten von der tatsächlichen Handlungsart des Probanden abweichen kann.


% evtl Straßenbefragungen machen??
% Jürgen & Simone?
\subsection{Auswertung}

Für die Auswertung der Umfrageergebnisse bieten sich verschiedene Ansätze.
Neben einem generellen Überblick ist vor allem der Vergleich zwischen Angehörigen der Berufsgruppen \ugspr{EDV/IT} und anderen aussagekräftig.

\subsubsection{Allgemeine Tendenzen}

Aus den Umfrageergebnissen lässt sich gut ableiten, welche Art von Täuschungsversuchen bei den Probanden am effizientesten ist.
Das angeführte Beispiel des Mitarbeiters mit dem Kuchenblech ist augenscheinlich besonders effizient.
Ca. 84\% aller Befragten würden eine solche Person in das Gebäude hereinlassen bzw. diesem sogar die Türe aufhalten.
Paradoxerweise ist das Ergebnis weniger eindeutig, wenn ein Mitarbeiter lediglich seinen ID-Chip vergessen hat.
Nur 33\% der Befragten geben in diesem Szenario an, den fremden Mitarbeiter einzulassen.
Die Gründe hierfür können verschiedene sein.
Um weiterführende Antworten zu finden ist es wichtig zu wissen, womit sich die Probanden in einer solchen Situation beschäftigen, welche Fragen bei diesen aufkommen und vor allem wie diese beantwortet werden.
Im zweiten Beispiel kommt durch die Feststellung des Offensichtlichen, dass die fremde Person keinen ID-Chip bei sich trägt, implizit die Frage auf, wo dieser ID-Chip sich befindet.
Durch seine bloßes Erscheinungsbild liefert der Angreifer per se keine passende Antwort.
Dies führt in vielen Fällen unwillkürlich zu der Frage, ob er oder sie tatsächlich eine Zugangskarte besitzt und zum Unternehmen gehört.
Der vermeintliche Kollege mit dem Kuchenblech beantwortet aufkommende Fragen und somit aufkeimendes Misstrauen implizit durch sein Auftreten.
Die Frage nach dem ID-Chip erübrigt sich, da der vermeintliche Mitarbeiter offensichtlich nicht in der Lage ist, seine Zugangskarte hervor zu holen bzw. dies nur mit großer Mühe möglich wäre.
Er appelliert durch sein Auftreten implizit an die Hilfsbereitschaft der anwesenden Personen und dreht damit die Situation gewissermaßen um.
Somit müssten die Anwesenden sich erklären, wenn sie der Person die Hilfe verwehren.

Sicherlich steht die Aktionsbereitschaft auch mit dem Gefahrenpotenzial des Angriffs in Verbindung.
Im eben aufgeführten Beispiel ist die Gefahr, dass eine unbefugte Person in das Gebäude eindringt.
Wesentlich heikler ist es, wenn fremde Geräte wie z.B. ein USB-Stick an einen Rechner innerhalb des Unternehmens angeschlossen werden.
Die Hemmschwelle hierfür ist bereits etwas stärker ausgeprägt, wenn auch nicht deutlich.
Etwa 54\% geben an der Person den Gefallen zu tun.

Sehr deutlich ist das Ergebnis im Bezug auf telefonische Angriffe (Phone Elicitation) durch eine Person des anderen Geschlechts.
Sowohl für Frauen als auch für Männer ist festzustellen, dass ein solcher Anruf schnell auf Mistrauen stößt.
Ausschließlich 16\% der Befragten gaben an Informationen wie z.B. die Mitarbeiternummer auf Anfrage herauszugeben.
Dies lässt sich unter anderem durch den fehlenden Pretext der anrufenden Person erklären.
Der angerufenen Person bieten sich wenige Möglichkeiten aufkommende Fragen zur Identität des Angreifers implizit zu beantworten.
Durch die geringe Persönlichkeit des Anrufs ist auch die Basislinie an Misstrauen bereits wesentlich höher als bei einem tatsächlichen Identitätsbetrug.

\begin{figure}[htbp]
	\centering
	\includegraphics[width=0.6\textwidth]{abb/vertrauen_email_phone.png}
	\caption{Vertrauen in bekannte E-Mail-Adressen und Telefonnummern}
	\label{fig:vertrauen-mail-phone}
\end{figure}

Überprüft man das Vertrauen in bekannte Mail-Adressen oder Telefonnummern kann man den Umfrageergebnissen eine deutliche Tendenz zum Vertrauen entnehmen.
Abbildung \ref{fig:vertrauen-mail-phone} macht deutlich, dass eine signifikante Mehrheit der Mail voll bis bedingt Vertraut.
Die Fragestellung selbst legt zwar noch keinen Fokus darauf, jedoch wird das Vertrauen in die Mail auch stark von deren Inhalt beeinflusst.
Somit erklärt sich auch dass die Antworten zwischen vollem und bedingtem Vertrauen in etwa gleich gewichtet sind.
Nebenbei ist im Ansatz zu erkennen, dass das Vertrauen in eine bekannte Rufnummer etwas stärker ausgeprägt ist als das in eine vertraute E-Mail-Adresse.
Dies kann jedoch im Rahmen der Standardabweichung liegen.
Um eine solche Hypothese zu untermauern müsste dies mit einer größeren Stichprobe überprüft werden.

Wenig überraschend ist die Aufmerksamkeit, welche bestimmten Personen-/Berufsgruppen im Alltag entgegengebracht wird.
Wie aus Abbildung \ref{fig:aufmerksamkeit} zu entnehmen erfährt eine Putzkraft verglichen mit einer elegant gekleideten Person (Führungskraft oder gehobener Angestellter) weniger Aufmerksamkeit.
Die Aufmerksamkeit, welche einem arbeitenden Techniker gilt, ist dazwischen vorzufinden.
Aus diesen Ergebnissen ist direkt kein Rückschluss auf Gefahren für ein Unternehmen möglich, jedoch spiegelt es das Risiko für den Social Engineer wider.
Je stärker der Angreifer im Fokus der Wahrnehmung anderer steht, desto glaubwürdiger muss sein Auftreten und somit auch die implizite Beantwortung von Fragen zur Identität sein.
Für den Angreifer stellt sich natürlich auch die Frage, was er erreichen möchte, denn jeder der drei im Beispiel aufgeführten Personengruppen stehen unterschiedliche Zugangsberechtigungen zur Verfügung.

\begin{figure}[htbp]
	\centering
	\includegraphics[width=0.6\textwidth]{abb/aufmerksamkeit_vergleich.png}
	\caption{Aufmerksamkeit gegenüber verschiedenen Berufsgruppen}
	\label{fig:aufmerksamkeit}
\end{figure}

Abschließend kann anhand des Szenarios in dem der Hausmeister einige Kisten aus dem Bürokomplex trägt überprüft werden, wie sich das Verhalten seitens der Mitarbeiter bei einer funktionierenden Identität des Angreifers darstellt.
Abbildung \ref{fig:hausmeister_allg} zeigt deutlich, dass die Mehrheit der Befragten gar keine Reaktion zeigen würde.
Nicht einmal 20\% tauschen sich über das Gesehene mit ihren Kollegen aus.
Aktiv werden lediglich 16\% von denen 15\% nachfragen, was der Hausmeister in den Kisten transportiert und nur 1\% tatsächlich die Identität des Hausmeisters überprüft.

\begin{figure}[htbp]
	\centering
	\includegraphics[width=0.6\textwidth]{abb/hausmeister_allg.png}
	\caption{Reaktion auf verdächtiges Verhalten einer funktionierenden Identität}
	\label{fig:hausmeister_allg}
\end{figure}

Zudem wurde das Wissen der befragten Personen zum Thema Social Engineering überprüft.
Dabei wurde gefragt, ob der Begriff \fachwort{Social Engineering} bekannt ist, die Person von Techniken wie Telefon-Spoofing weiß oder ob ihr klar ist, dass der angezeigte Linkname von der tatsächlichen Referenz abweichen kann.
Anhand der Ergebnisse, wie sie in Abbildung \ref{fig:social_engineering_wissen} zu sehen sind, wird ersichtlich, dass die Mehrheit von diesen Techniken weiß.
Dass dies jedoch vor Angriffen schützt ist in keiner Weise garantiert.

\begin{figure}[htbp]
	\centering
	\includegraphics[width=0.6\textwidth]{abb/bekanntheit_allgemein.png}
	\caption{Wissen über Social Engineering und Techniken}
	\label{fig:social_engineering_wissen}
\end{figure}

An dieser Stelle sei noch einmal betont, dass es sich hierbei um eine allgemeine Auswertung handelt.
Ob sich zwischen verschiedenen Berufsgruppen die Wahrnehmungsverteilung ändert, ist zunächst nicht abzuleiten.
Das nächste Kapitel zeigt Unterschiede zwischen den verschiedenen Berufsgruppen auf.
% allgemeine Wahrnehmung von Social Engineering

% an dieser Stelle sollten noch einige Schwachstellen des Tests dargelegt werden.


\subsubsection{Berufsgruppenorientierte Auswertung}\label{ref:berufsgruppenorientierte-auswertung}

Von besonderem Interesse ist vor allem der Vergleich zwischen der Berufsgruppe der EDV/IT-Mitarbeiter mit IT-fremden Berufsgruppen.
Im Vergleich zu anderen Berufsbildern haben Mitarbeiter der EDV/IT meist indirekt oder auch direkt mit der Sicherheit der Daten zu tun.
Oft ist für den Datenschutz ein Team aus der EDV-Abteilung verantwortlich.
Somit lässt sich vermuten, dass solche Personen sensibler für bestimmte Angriffstechniken sind und grundsätzlich wachsamer und informierter sind.
Bei vielen Szenarien und Fragestellungen können jedoch keine signifikanten Unterschiede festgestellt werden.
Das Vertrauen in bekannte E-Mail-Adressen und Rufnummern ist im Vergleich zur gesamten Stichprobe ähnlich verteilt.
Ebenso gestaltet sich die Wahrnehmung verschiedener anderer Berufsgruppen übereinstimmend.

Kleinere Unterschiede ergeben sich unter anderem bei der Auswertung der Handlungsmuster.
Während das Verhalten auf den Mitarbeiter mit dem Kuchenblech und dem ohne ID-Chip dem der allgemeinen Auswertung ähnelt, ist ein Unterschied bei dem Szenario mit dem USB-Stick einer unbekannten Person zu beobachten.
Während Angehörige anderer Berufsgruppen zu ca. 60\% dieser Person den Gefallen erweisen würden, sind unter den EDV-Mitarbeitern der Probanden lediglich 40\% zu finden.
Dies lässt sich sicherlich auf die Aufgeklärtheit des Personenkreises über die Gefahren eines harmlos erscheinenden USB-Speichers zurückführen.
Nichtsdestotrotz birgt eine Erfolgsrate von 40\% (aus der Sicht eines Angreifers) immer noch ein nicht zu unterschätzendes Potenzial für Angriffe.

\subsubsection{Altersorientierte Auswertung}\label{ref:altersorientierte-auswertung}

Das Alter betreffend wären verschiedene Argumentationen denkbar.
Zum einen ließe sich behaupten altere Mitarbeiter, d.h. 30 Jahre oder älter, seien gegenüber Social Engineering Angriffen besser gewappnet, da sie mit mehr Erfahrung ausgestattet sind und zudem weniger zu leichtsinnigen oder unüberlegten Handlungen tendieren.
Dem entgegensetzen könnte man die Tatsache, dass die aus dieser Erfahrung resultierende Routine bei geschickt gestellten Anweisungen einem Social Engineer zu seinem Erfolg verhelfen kann.
In Anbetracht der zugrunde liegenden Umfrageergebnisse kann jedoch keine dieser beiden Thesen untermauert werden.
Zwar sind minimale Abweichungen innerhalb der Szenarien mit dem Mitarbeiter mit Kuchenblech, dem USB-Stick und dem fehlenden ID-Chip zu sehen, jedoch besitzen diese bei weitem keine signifikanten Differenzen.

Bei einem Blick auf die Fragen bezüglich der Wahrnehmung gegenüber anderen Berufsgruppen ist bei den jüngeren Probanden eine Tendenz zu erkennen, Putzkräften überdurchschnittlich wenig Beachtung zu schenken.
Dies könnte ein Anhaltspunkt für einen Social Engineer sein, der z.B. vor der Aufgabe steht die Räumlichkeiten eines Unternehmens mit vielen jungen Mitarbeitern auszukundschaften.
Eine Ähnliche Tendenz ist bei der Berufsgruppe der Techniker zu erkennen.

Das Wissen um die technischen Möglichkeiten eine falsche Identität erfolgreich vorzutäuschen ist in den Köpfen der jungen Probanden präsenter.
Dies lässt sich durch die geringere Distanz zu technischen Neuerungen und Möglichkeiten im Allgemeinen erklären.
Dies wiederum könnte einen Angreifer zu der Überlegung bringen ein Unternehmen mit einem hohen Durchschnittsalter der Mitarbeiter per Phone Elicitation oder mittels gespoofter E-Mail-Adressen anzugreifen.

Am Rande sei noch erwähnt, dass junge Mitarbeiter nach eigenen Angaben Entscheidungen von ihnen nicht persönlich bekannten Führungspersonen kritischer Beäugen als ihre älteren Kollegen.
Abbildung \ref{fig:kritische-beurteilung} zeigt deutlich, dass nur eine kleine Minderheit die Entscheidungen eines Vorgesetzten unkritisch entgegen nimmt.
Der Großteil steht dieser Entscheidung misstrauisch bis neutral gegenüber.
Bei älteren Mitarbeitern ist diese Ausprägung weniger deutlich.

\begin{figure}[htbp]
	\centering
	\includegraphics[width=0.6\textwidth]{abb/altersgruppen-vertrauen-fuehrungsperson.png}
	\caption{Kritische Beurteilung von Entscheidungen unbekannter Führungspersonen}
	\label{fig:kritische-beurteilung}
\end{figure}

\subsubsection{Unternehmensbezogene Auswertung}\label{sec:unternehmensbezogene-auswertung}

Unternehmen seien an dieser Stelle anhand zweier Ausprägungen zu unterscheiden.
Zum einen lassen sich Unternehmen auf ihre Größe bezogen unterscheiden (messbar durch die Zahl der Mitarbeiter), zum anderen ist die Ausdehnung des Unternehmens (messbar durch die Zahl der Standorte) ein wichtiges Merkmal.
Um ein abgerundetes Bild der Umfrage zu erhalten, ist es wichtig neben den bisher aufgeführten Aspekten auch die unternehmensspezifischen Auswirkungen zu untersuchen.

\begin{figure}[htbp]
	\centering
	\includegraphics[width=0.6\textwidth]{abb/id_chip-unternehemsgroesse.png}
	\caption{Reaktionen bezogen auf die Unternehmensgröße}
	\label{fig:id-chip-unternehmensgroesse}
\end{figure}

Größtenteils sind die Tendenzen zwischen verschiedenen Unternehmensgrößen identisch und decken sich mit den bisher gewonnenen Erkenntnissen.
Jedoch kann eine leichte Steigerung der Wachsamkeit mit wachsender Unternehmensgröße festgestellt werden.
Abbildung \ref{fig:id-chip-unternehmensgroesse} zeigt dies anhand des ID-Chip-Szenarios.
Zwar neigen kleine, mittelständische und große Unternehmen alle dazu den vermeintlichen Mitarbeiter nicht ohne weiteres passieren zu lassen, jedoch steigt diese Ausprägung mit wachsender Mitarbeiterzahl.


Setzt man die Ergebnisse mit der Menge der Standorte in Relation, so lassen sich ähnliche Beobachtungen machen.
Eine mögliche Erklärung hierfür ist, dass es sich bei großen und verbreiteten Unternehmen um etablierte Konzerne handelt, bei denen Informationssicherheit und Datenschutz einen enormen Stellenwert besitzt.
Kleine oder mittelständische Unternehmen hingegen haben in der Regel noch kein ausgereiftes und vor allem kein ganzheitliches Sicherheitskonzept in Betrieb.
Bei Großkonzernen sind viele der später in Kapitel \ref{sec:hardening} vorgestellten Gegenmaßnahmen bereits implementiert.


\subsection{Anhaltspunkte für weitere Untersuchungen}\label{sec:anhaltspunkte-fuer-weitere-untersuchungen}

Die aus der Umfrage gewonnenen Erkenntnisse geben bereits ein gutes allgemeines Bild wieder.
Auch lassen sich anhand der spezifischen Auswertungen bereits genauere Rückschlüsse ziehen.
In den auswertenden Kapiteln war jedoch zu erkennen, dass diese Untersuchungen lediglich die Oberfläche einiger Themen tangieren.
Ein interessanter Anhaltspunkt für weitere Untersuchungen sind genauere Analysen einzelner Personengruppen.
Eine detaillierte Übersicht von verschiedenen Altersgruppen könnte Ausgangspunkt für eine weitere Untersuchung sein.
Die Fragestellungen und Szenarien selbst können ebenfalls stärker fokussiert werden.
Hierfür bietet es sich an Einbruchstechniken wie das Szenario mit dem Kuchen in vielen abgewandelten Formen zu überprüfen.
Schwerpunkte können jedoch auch auf Themen wie die Wirkung der Körpersprache, einer E-Mail oder bestimmter Kleidung gelegt werden.

