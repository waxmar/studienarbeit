\section{Umfrage}\label{sec:umfrage}
Bis hierhin ist geschildert worden, worum es sich bei Social Engineering handelt, welche Techniken dabei zum Einsatz kommen und wie das damit verbundene Risiko einzuschätzen ist.
Um die tatsächlichen Anfälligkeiten zu überprüfen ist im Rahmen der Studienarbeit eine Umfrage durchgeführt worden, bei welcher verschiedene berufstätige Personen zu verschiedenen Themen befragt wurden.
Die Umfrage mit dem Titel \ugspr{Hilfsbereitschaft und Social Engineering} und die daraus gewonnenen Ergebnisse werden im folgenden vorgestellt, aufbereitet und diskutiert.
Ebenfalls wird die Vorgehensweise skizziert.

\subsection{Fragebogen}
Für die Verteilung und Erstellung wurde das Web bzw. die Software Limesurvey verwendet.
Die Umfrageergebnisse sind mittels einiger demographischer Daten kategorisiert.
Bei den für diese Umfrage erhobenen Daten handelt es sich um das Alter, das Geschlecht, die Berufsgruppe, die Unternehmensgröße sowie die Anzahl der Standorte.
Im weiteren Verlauf werden Fragen zu verschiedenen Themenschwerpunkten gestellt.
Dabei handelt es sich konkret um Fragen zu Hilfsbereitschaft, Vertrauen, Aufmerksamkeit und einigen abschließenden Fragen zu Social Engineering selbst.

Im ersten Abschnitt (Hilfsbereitschaft) werden Fragen zu Situationen im Arbeitsalltag gestellt.
Diese Szenarien sind maßgeblich an den in Kapitel \ref{sec:gangige_angriffe} vorgestellten Angriffsmustern orientiert und sollen die Anfälligkeit der Probanden nachweisen.
Bei den Fragen handelt es sich um Ja/Nein-Fragen, bei denen angegeben werden soll, ob man dazu tendiert die erwartete Handlung auszuführen oder diese zu unterlassen.
Bei den ersten drei Fragen handelt es sich um direkte Angriffe bei denen eine gesamte Identität vorgetäuscht wird, die letzte Frage bezieht sich auf die Technik Phone Elicitation.
Zudem überprüft die letzte Frage die Mitteilungsbereitschaft gegenüber den anderen Geschlecht.

Die darauf folgende Serie von insgesamt sechs Fragen bezieht sich auf das Vertrauen der Probanden.
Dabei gibt es Fragen zu allen der drei Angriffskategorien: Identitätsbetrug, Phone Elicitation und E-Mail-Phishing.
Dabei wird das Vertrauen in die Identität von angezeigten Telefonnummern und E-Mail-Adressen (und somit indirekt das bewusste Wissen über Spoofing-Techniken) überprüft.
Damit ist die Problematik gemeint, dass eine Person tatsächlich über dieses Thema informiert ist, in der konkreten Situation jedoch nicht daran denkt.
Die letzten beiden Fragen zeigen Bilder von jeweils einer Person und sollen prüfen inwiefern die Optik das Vertrauen beeinflusst.
Innerhalb dieser Serie sollen Bewertungen des Vertrauens auf einer Skala von eins bis fünf angegeben werden.

In der folgenden Kategorie werden dem Probanden drei Bilder gezeigt, welche jeweils verschiedene Personen darstellen.
Zum einen wird eine Reinigungskraft gezeigt, welche gerade einen Fußboden wischt, zum anderen ein Techniker, der gerade an einer Art Verteilerkasten steht sowie eine Führungskraft.
Dabei soll der Proband angeben wie sehr er diese Personen im Arbeitsalltag wahrnimmt.
Wie in der vorangegangen Kategorie wird dies mittels einer Skala von eins bis fünf festgestellt.
Abschließend werden Fragen zu Social Engineering gestellt.
Der Proband wird dabei gefragt, ob er von Techniken wie Spoofing bereits gehört hat.
Außerdem soll er eine persönliche Einschätzung des Risikos abgeben.

Die Umfrage selbst ist für den Zeitraum von ca. sechs Wochen öffentlich verfügbar gewesen und ist von freiwilligen Probanden durchgeführt worden.
Verteilt wurde die Umfrage weitestgehend mittels sozialer Netze.

% Bezüge auf psychologische Grundlagen machen

\subsection{Nachträgliche Betrachtung}
Im Rückblick vor allem jedoch durch die Betrachtung der Ergebnisse sind einige Punkte verdeutlicht worden, welche dem Fragebogen Aussagekraft entziehen bzw. Potenzial für weitere Aussagen entzogen haben.
Innerhalb der demographischen Daten ist es wie den Ergebnissen des Fragebogens zu entnehmen ist, leider nicht gelungen, die Berufsgruppen sinnvoll abzudecken.
Zwar können Kernaussagen darüber getroffen werden, inwiefern sich IT-affine Berufsgruppen von solchen Unterscheiden, die sich nicht näher damit beschäftigen, jedoch ist innerhalb letzterer keine Möglichkeit gegeben genauere Erkenntnisse zu gewinnen.
Eine granulare Aufteilung ist an dieser Stelle sicherlich interessant und würde für verschiedene Bereiche in Unternehmen einen sinnvollen Aufschluss darüber geben, welche Abteilungen leichter anzugreifen sind als andere.
Auch sind andere Merkmale gar nicht erfasst worden wie z.B. die Dauer der Berufstätigkeit oder wie viele verschiedene Berufsgruppen an einem Standort vertreten sind. % Merkmale nennen

Die Fragen, welche Personen zeigen, haben seitens der Auswertung ein großes Potenzial, welches aufgrund der begrenzten Umfragedauer nicht vollständig ausgeschöpft werden konnte.
So könnte man zur Fragegruppe Wahrnehmung wesentlich detailliertere Erkenntnisse gewinnen.
Gerade in Kombination mit genaueren Angaben im Bereich der Berufsgruppe können hierbei sicherlich nützliche Hinweise herausgearbeitet werden.

Alldem ist hinzuzufügen, dass es sich bei den Ergebnissen letztlich nicht um eine Überprüfung des Ernstfalls handelt sondern lediglich eine theoretische Befragung vorliegt.
Unterschiede zu einer Situation wie sie tatsächlich vorkommt liegen vor allem in der persönlichen Distanz der Person.
Wird man auf einem Fragebogen mit einer solchen Situation konfrontiert ist es wesentlich besser möglich sich davon zu distanzieren und klarer darüber nachzudenken.
Zudem ist dem Probanden dadurch mehr Zeit gegeben.
Zwar sind die Probanden zum einen darauf hingewiesen worden, dass es sich um ein anonyme Umfrage handelt, zum anderen wurde darum gebeten die Fragen so zu beantworten, dass es dem eigenen Verhalten entspricht, jedoch ist nicht auszuschließen, dass ein kleiner Anteil der Anworten von der tatsächlichen Handlungsart des Probanden abweichen kann.


% evtl Straßenbefragungen machen??
% Jürgen & Simone?
\subsection{Auswertung}

Für die Auswertung der Umfrageergebnisse bieten sich verschiedene Ansätze.
Neben einem generellen Überblick ist vor allem der Vergleich zwischen Angehörigen der Berufsgruppen \ugspr{EDV/IT} und anderen aussagekräftig.

\subsubsection{Allgemeine Tendenzen}

Aus den Umfrageergebnissen lässt sich gut ableiten, welche Art von Täuschungsversuchen bei den Probanden am effizientesten ist.
Das angeführte Beispiel des Mitarbeiters mit dem Kuchenblech ist augenscheinlich besonders effizient.
Ca. 84\% aller Befragten würden eine solche Person in das Gebäude hereinlassen bzw. diesem sogar die Türe aufhalten.
Paradoxerweise ist das Ergebnis weniger eindeutig, wenn ein Mitarbeiter lediglich seinen ID-Chip vergessen hat.
Nur 33\% der Befragten geben in diesem Szenario an, den fremden Mitarbeiter einzulassen.
Die Gründe hierfür können verschiedene sein.
Um weiterführende Antworten zu finden ist es wichtig zu wissen, womit sich die Probanden in einer solchen Situation beschäftigen, welche Fragen bei diesen aufkommen und vor allem wie diese beantwortet werden.
Im zweiten Beispiel kommt durch die Feststellung des Offensichtlichen, dass die fremde Person keinen ID-Chip bei sich trägt, implizit die Frage auf, wo dieser ID-Chip sich befindet.
Durch seine bloßes Erscheinungsbild liefert der Angreifer per se keine passende Antwort.
Dies führt in vielen Fällen unwillkürlich zu der Frage, ob er oder sie tatsächlich eine Zugangskarte besitzt und zum Unternehmen gehört.
Der vermeintliche Kollege mit dem Kuchenblech beantwortet aufkommende Fragen und somit aufkeimendes Misstrauen implizit durch sein Auftreten.
Die Frage nach dem ID-Chip erübrigt sich, da der vermeintliche Mitarbeiter offensichtlich nicht in der Lage ist, seine Zugangskarte hervor zu holen bzw. dies nur mit großer Mühe möglich wäre.
Er appelliert durch sein Auftreten implizit an der Hilfsbereitschaft der anwesenden Personen und dreht damit die Situation gewissermaßen um.
Somit müssten die Anwesenden sich erklären, wenn sie der Person die Hilfe verwehren.

Sicherlich steht die Aktionsbereitschaft auch mit dem Gefahrenpotenzial des Angriffs in Verbindung.
Im eben aufgeführten Beispiel ist die Gefahr, dass eine unbefugte Person in das Gebäude eindringt.
Wesentlich heikler ist es, wenn fremde Geräte wie z.B. ein USB-Stick an einen Rechner innerhalb des Unternehmens angeschlossen werden.
Die Hemmschwelle ist hierfür ist bereits etwas stärker ausgeprägt, wenn auch nicht deutlich.
Etwa 54\% geben an der Person den Gefallen zu tun.

Sehr deutlich ist das Ergebnis im Bezug auf telefonische Angriffe (Phone Elicitation) durch eine Person des anderen Geschlechts.
Sowohl für Frauen als auch für Männer ist festzustellen, dass ein solcher Anruf schnell auf Mistrauen stößt.
Ausschließlich 16\% der Befragten gaben an Informationen wie z.B. die Mitarbeiternummer auf Anfrage herauszugeben.
Dies lässt sich unter anderem durch den fehlenden Pretext der anrufenden Person erklären.
Der angerufenen Person bieten sich wenige Möglichkeiten aufkommende Fragen zur Identität des Angreifers implizit zu beantworten.
Durch die geringe Persönlichkeit des Anrufs ist auch die Basislinie an Misstrauen bereits wesentlich höher als bei einem tatsächlichen Identitätsbetrug.


% an dieser Stelle sollten noch einige Schwachstellen des Tests dargelegt werden.