\section{Einleitung}\label{sec:2_einleitung}
\fachwort{Social Engineering} könnte man im Allgemeinen mit \ugspr{gesellschaftliches Ingeniuerwesen} übersetzen.
Jedoch trifft es die Bezeichnung \speech{angewandte Sozialwissenschaften} wesentlich besser.
Die Techniken hinter \fachwort{Social Engineering} reichen schon sehr weit in die Vergangenheit zurück, wohingegen der Begriff selbst erst seit einigen Jahren durch verschiedene Hacker wie z.B. \name{Kevin Mtnick} geprägt wurde.
Worum es sich bei \fachwort{Social Engineering} handelt soll im Rahmen dieser Ausführung in erster Linie geklärt werden.

Vielmehr stellt sich allerdings die Frage, wie \fachwort{Social Engineering} von der IT oder anderen Abteilungen gesehen und eingestuft wird.
Da es sich um einen recht neuen Terminus handelt, ist dieser womöglich noch nicht vollends in den Unternehmen und Einrichtungen angekommen.
Zwar kennt ein jeder die Rundmails eines nigerianischen Prinzen, der für sein Lebensglück noch schnell ein paar tausend Euro benötigt, jedoch handelt es sich bei solchen Maschen nur um die Spitze des Eisbergs \fachwort{Social Engineering} - die tatsächlichen Möglichkeiten und Gefahren reichen viel weiter.

Um dieser Frage nachzugehen wird das Thema sowohl aus der Sicht des Angreifers als auch aus der Sicht der potenziellen Zielperson analysiert und ausgearbeitet.
Dabei entpuppt sich \fachwort{Social Engineering} als Vielkampf der angewandten Wissenschaften.
Für einen Social Engineer gilt es oft die Bereiche Soziologie, Psychologie und einem gewissen Grad an schauspielerischem Können mit technischem Know-How zu verknüpfen.

\subsection{Ziel der Arbeit}\label{sec:ziel_der_arbeit}
Im Rahmen dieser Arbeit sollen einige Aspekte zum Thema Social Engineering überprüft und näher untersucht werden.
Der Kern der Ausarbeitung liegt im Ergebnis einer Umfrage.
Anhand dieser soll zunächst festgestellt werden, wie stark die Gefahren solcher Angriffe wahrgenommen werden.
Dabei gilt es außerdem herauszufinden, ob das Bewusstsein der Probanden für Social Engineering kontextabhängig ist.
Außerdem soll untersucht werden, ob es diesbezüglich Auffälligkeiten gibt, die darauf hindeuten, dass bestimmte Personengruppen stärker und weniger anfällig sind.
Ziel dieser Betrachtung ist es herauszufinden, ob bestimmte Personengruppen schlecht gegen Social Engineering Angriffe gewappnet sind, tatsächlich der Meinung sind für solche Angriffe keinen Nährboden bieten zu können.
Des weiteren sollen Korrelationen zwischen Anfälligkeit und anderen Attributen wie Alter, Geschlecht oder ähnlichem hergestellt werden.
Das Ziel dieser Ausarbeitung ist es, ein grundlegendes Verständnis von Social Engineering zu vermitteln
und anhand der Untersuchungen die Gefährlichkeit einzustufen. Zudem sollen in Anbetracht der Ergebnisse
Schutzmöglichkeiten eruiert werden.

Diese Ausarbeitung behandelt jedoch nicht die konkrete Vorgehensweise von Social Engineers.
D.h. es werden weder Tools analysiert noch werden detaillierte Fallanalysen durchgeführt.
Es werden lediglich dem Verständnis förderliche Beispiele erläutert.
Zudem sollen die beiden Kategorien telefonische und persönliche (direkte) Attacke nur theoretisch behandelt werden.
Da die Arbeit ebenso auf zahlreiche psychologische Grundlagen zurückgreift, werden Themen wie \fachwort{Nonverbale} Kommunikation ebenso kurz aufgegriffen.
 	
% keine genaue erklärung was in seminaren unterrichtet wird! lediglich orientierung an psych grundlagen
% keine Beschreibung technischer Grundlagen für Spoofing etc.

\subsection{Vorgehensweise}\label{sec:vorgehensweise}
Zu Beginn der Arbeit steht eine umfangreiche Grundlagenrecherche.
Dafür werden die Themen Vertrauen, Manipulation, Kommunikation, Risikomanagement und natürlich Social Engineering selbst analysiert.
Aus den Ergebnissen dieser Recherche wird zum einen ein Fragebogen entworfen, welcher die Empfänglichkeit für Social Enginneering überprüfen soll, und zum anderen wird eine Phishing-Mail konstruiert.
Dabei sollen aus der Theorie gewonnene Erkenntnisse vertieft und in der Anwendung geprüft werden.

Für die Recherche des Themas \fachwort{Social Engineering} wird Literatur von auf diesem Gebiet wegweisenden Experten zu Rate gezogen. Darunter fallen die Autoren \name{Kevin Mitnick}, \name{Christopher Hadnagy} und \name{Ian Mann}.
Das Thema \ugspr{Vertrauen}, bei dem es sich um ein psychologisches wie soziologisches Phänomen handelt, wird zu Großen Teilen anhand der Autoren \name{Niklas Luhmann} und \name{Bruce Schneier} aufgearbeitet.
Ausgangspunkt für die Recherche zum Thema Kommunikation und Kommunikationsmodelle bietet die Arbeit \speech{Grundlagen der Kommunikation} von \name{Markus Plate}, welche unter anderem auf die grundlegenden Ergebnisse von \name{Paul Watzlawick} aufbauen.
Das Werk \speech{Die Psychologie des Überzeugens} von \name{Robert Cialdini} sowie die Arbeiten von \name{Hadnagy} bieten die Basis des Kapitels zum Thema \fachwort{Manipulation}.
\fachwort{Risikomanagemt} wird anhand der Autoren \name{Dan Borge} und \name{Ian Mann} näher erläutert.

\subsection{Leitfaden}\label{sec:aufbau_der_arbeit}
\Kapitel{sec:social_engineering} stellt den Begriff Social Engineering und die Idee dahinter vor.
Dabei werden auch Beispiele für mögliche Angriffe gegeben.

In Kapitel \Kapitel{sec:risikoanalyse} wird der Begriff Risiko näher erläutert.
Da es sich bei Social Engineering ebenfalls um ein Risiko handelt werden die Analyse und die Handhabung
von Risiken sowie damit verbundene Schwierigkeiten genannt.

Um Social Engineering zu verstehen, ist es wichtig die zu Grunde liegenden Prinzipien zu erläutern.
In \Kapitel{sec:psychologische_grundlagen} werden einige Verhaltensmuster vorgestellt und aufgezeigt
wie diese gezielt ausgenutzt werden können.
Zentrale Themen sind hierbei vor allem Vertrauen, das Zusammenspiel von Bewusstsein und Unterbewusstsein.

Kapitel \Kapitel{sec:umfrageanalyse} befasst sich mit der Auswertung einer Umfrage, die im Rahmen der
Studienarbeit durchgeführt worden ist.

Im darauf folgenden \Kapitel{sec:praxis} wird die praktische Anwendung einer Social Engineering Attacke 
