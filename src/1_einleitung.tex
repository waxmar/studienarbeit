\section{Einleitung}\label{sec:2_einleitung}
Diese Arbeit behandelt das Thema Social Engineering.

\subsection{Ziel der Arbeit}\label{sec:ziel_der_arbeit}
Im Rahmen dieser Arbeit sollen einige Aspekte zum Thema Social Engineering überprüft und näher untersucht werden.
Der Kern der Ausarbeitung liegt im Ergebnis zweier Fragebögen.
Anhand dieser soll zunächst festgestellt werden, wie sehr die Gefahren solcher Angriffe wahrgenommen werden.
Dabei gilt es außerdem herauszufinden, ob das Bewusstsein der Probanden für Social Engineering kontextabhängig ist.
Außerdem soll untersucht werden, ob es diesbezüglich Auffälligkeiten gibt, die darauf hindeuten, dass bestimmte Personengruppen stärker und weniger anfällig sind.
Dem soll zudem die Selbstsicherheit gegenübergestellt werden.
Ziel dieser Betrachtung ist es herauszufinden, ob bestimmte Personengruppen schlecht gegen Social Engineering Angriffe gewappnet sind, tatsächlich der Meinung sind für solche Angriffe keinen Nährboden bieten zu können.
Damit soll neben der eigentlichen Analyse der Risikogruppen auch unbelegte Klischees bestätigt bzw. widerlegt werden.
Des weiteren sollen auch andere Korrelationen zwischen Anfälligkeit und anderen Attributen (sofern vorhanden) hergestellt werden.

Zusätzlich zur Analyse der Personengruppen, sollen die Reaktionen auf eine reale Methode getestet werden.
Konkret geht es dabei um eine Variante des E-Mail-Phishing.

Das Ziel dieser Ausarbeitung ist es, ein grundlegendes Verständnis von Social Engineering zu vermitteln
und anhand der Untersuchungen die Gefährlichkeit einzustufen. Zudem sollen in Anbetracht der Ergebnisse
Schutzmöglichkeiten eruiert werden.

% was nicht Teil der Arbeit ist!
%	konkrete Vorgehensweise von Social Engineers (Tools etc.)
%	theoretischer Schwerpunkt zwar auf Psychologie
%		allerdings keine Vertiefung von NVC (Mimik, etc.) -> nur erwähnen, dass es das gibt
%	Telefon Attacken nur theoretisch
%	keine Durchführung direkter SE-Techniken
%		rein theoretisch
%	Keine detaillierten Fallanalysen (lediglich dem Verständnis förderliche Beispiele)
 	

\subsection{Vorgehensweise}\label{sec:vorgehensweise}
blabla

\subsection{Aufbau der Arbeit}\label{sec:aufbau_der_arbeit}
\Kapitel{sec:social_engineering} stellt den Begriff Social Engineering und die Idee dahinter vor.
Dabei werden auch Beispiele für mögliche Angriffe gegeben.

In Kapitel \Kapitel{sec:risikoanalyse} wird der Begriff Risiko näher erläutert.
Da es sich bei Social Engineering ebenfalls um ein Risiko handelt werden die Analyse und die Handhabung
von Risiken sowie damit verbundene Schwierigkeiten genannt.

Um Social Engineering zu verstehen, ist es wichtig die zu Grunde liegenden Prinzipien zu erläutern.
In \Kapitel{sec:psychologische_grundlagen} werden einige Verhaltensmuster vorgestellt und aufgezeigt
wie diese gezielt ausgenutzt werden können.
Zentrale Themen sind hierbei vor allem Vertrauen, das Zusammenspiel von Bewusstsein und Unterbewusstsein.

Kapitel \Kapitel{sec:umfrageanalyse} befasst sich mit der Auswertung einer Umfrage, die im Rahmen der
Studienarbeit durchgeführt worden ist.

Im darauf folgenden \Kapitel{sec:praxis} wird die praktische Anwendung einer Social Engineering Attacke 