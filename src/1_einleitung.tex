\section{Einleitung}\label{sec:2_einleitung}
Diese Arbeit behandelt das Thema Social Engineering.

\subsection{Ziel der Arbeit}\label{sec:ziel_der_arbeit}
Im Rahmen dieser Arbeit sollen einigie Aspekte zum Thema Social Engineering überprüft und näher untersucht werden.
Der Kern der Ausarbeitung liegt im Ergebnis zweier Fragebögen.
Anhand dieser soll zunächst festgestellt werden, wie sehr die Gefahren solcher Angriffe wahrgenommen werden.
Dabei gilt es außerdem herauszufinden, ob das Bewusstsein der Probanden für Social Engineering kontextabhängig ist.
Außerdem soll untersucht werden, ob es diesbezüglich Auffälligkeiten gibt, die darauf hindeuten, dass bestimmte Personengruppen stärker und weniger anfällig sind.
Dem soll zudem die Selbstsicherheit gegenübergestellt werden.
Ziel dieser Betrachtung ist es herauszufinden, ob bestimmte Personengruppen schlecht gegen Social Engineering Angriffe gewappnet sind, tatsächlich aber der Meinung sind für solche Angriffe keinen Nährboden bieten zu können.
Damit soll neben der eigentlichen Analyse der Risikogruppen auch unbelegt Klischees bestätigt bzw. widerlegt werden.
Desweiteren sollen auch andere Korrelationen zwischen Anfälligkeit und anderen Attributen (sofern vorhanden) hergestellt werden.


\subsection{Aufbau der Arbeit}\label{sec:aufbau_der_arbeit}
\Kapitel{sec:social_engineering} stellt den Begriff Social Engineering und die Idee dahinter vor.
Dabei werden auch Beispiele für mögliche Angriffe gegeben.

In Kapitel \Kapitel{sec:risikoanalyse} wird der Begriff Risiko näher erläutert.
Da es sich bei Social Engineering ebenfalls um ein Risiko handelt werden die Analyse und die Handhabung
von Risiken sowie damit Verbundene Schwierigkeiten genannt.

Die psychologischen Grundlagen zum Verständnis von Social Engineering werden in
\Kapitel{sec:psychologische_grundlagen} erklärt.
Dabei sind verschiedene Kommunikationsmodelle, Neuro-Linguistische Programmierung, Transaktionsanalyse
und das Zusammenspiel von Bewusstsein und Unterbewusstsein zentrale Themen.

Kapitel 5 stellt einige manipulative Techniken zur Anwendung vor.