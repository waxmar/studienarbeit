\section{Einleitung}\label{2_kap1}
Diese Arbeit behandelt das Thema Social Engineering.

\subsection{Aufbau der Arbeit}
Kapitel 2 stellt den Begriff Social Engineering und die Idee dahinter vor.
Dabei werden auch Beispiele für mögliche Angriffe gegeben.

In Kapitel 3 wird der Begriff Risiko näher erläutert.
Da es sich bei Social Engineering ebenfalls um ein Risiko handelt werden die Analyse und die Handhabung
von Risiken sowie damit Verbundene Schwierigkeiten genannt.

Die psychologischen Grundlagen zum Verständnis von Social Engineering werden in Kapitel 4 erklärt.
Dabei sind verschiedene Kommunikationsmodelle, Neuro-Linguistische Programmierung, Transaktionsanalyse
und das Zusammenspiel von Bewusstsein und Unterbewusstsein zentrale Themen.

Kapitel 5 stellt einige manipulative Techniken zur Anwendung vor.