\section{Social Engineering}\label{sec:socialengineering}
Dieses Kapitel beschreibt den Begriff Social Engineering und grenzt die dazu gehörenden Techniken von üblichen Vorgehensweisen des Hackens ab. Dabei wird in Kapitel 2.1 zunächst ein Definition gegeben. Im darauf folgenden Kapitel wird die Einfachheit solche Angriffe gegenüber technischer Angriffe hervorgehoben. Kapitel 2.3 zählt die von Social Engineering herrührenden Risiken für Unternehmen mit kritischen Daten auf und bewertet diese. Zum Schluss (Kapitel 2.4) werden Beispiele für verschiedene Methoden gegeben.

\subsection{Definition}\label{sec:definition}
Um den Begriff Social Engineering korrekt einordnen zu können müssen zunächst herkömmliche Aspekte der Informationssicherheit betrachtet werden. Bei diesen handelt es sich zum einen um physikalische Zugriffskontrolle (z.B. Identitätsprüfungen an Türen)und zum anderen um IT-Sicherheit (meistens wird allerdings bei IT-Sicherheit lediglich von „Sicherheit“ gesprochen). Diese beiden Sicherheitsaspekte haben zweifellos ihren Platz und ihre Berechtigung. In vielen Fällen werden aber nur diese Art Angriffe berücksichtigt, denen solche Systeme entgegenarbeiten sollen. Offensichtlich ist, dass sich IT-Sicherheit und physikalische Sicherheit ausschließlich auf ein Unternehmen beschränken, welches lediglich aus IT-Systemen und Gebäuden mit Türen und Fenstern bestehen. Eine der essenziellsten Kernkomponenten des Unternehmens wird dabei gänzlich übersehen. Der Mitarbeiter stellt das größte Kapital des Unternehmens dar. Durch Social Engineering Techniken wird er jedoch zugleich zur größten Sicherheitslücke eines Unternehmens. Dies liegt größtenteils an den mangelnden Gegenmaßnahmen die es zu Social Engineering Attacken gibt.

Nun stellt sich weiter die Frage, was einen solchen Angriff ausmacht. Social Engineering Attacken zielen darauf ab, bestimmte Personen dahingehend zu manipulieren bestimmte Informationen herauszugeben oder Handlungen auszuführen, für die der EIngreifer selbst keine Berechtigung besitzt. Um solche Handlung auszulösen werden verschiedenste Techiken zur Täuschung herangezogen, die alle auf psychologischen Erkenntnissen beruhen. Auf diese wird in Kapitel 3 Psychologische Grundlagen tiefer eingegangen.

\subsection{Alternative zu technischen Methoden}\label{alternativezutechnischenmethoden}
In den letzten Jahren erfreuten sich Social Engineering Angriffe zunehmender Beliebtheit. Diese Entwicklung ist nicht ohne Grund zu beobachten. Während seit Beginn des Informationszeitalters auch die Angriffe auf Datenbestände immer häufiger und vor allem gefährlicher geworden sind, wurden entsprechend starke Gegenmaßnahmen entwickelt. Diese beschränken sich bis heute auf die Apsekte der physikalischen Sicherheit und der IT-Sicherheit. Auch heute noch spielt vor allem die IT-Sicherheit in vielen Unternehmen sicherlich gerechtfertigt eine übergeordnete Rolle. In jedem Unternehmen finden sehr wirkungsvolle aber auch ebenso kostenintensive Abwehrmechanismen ihre Anwendung. Diese stellen zwar kein unüberwindbares Hindernis dar, halten aber dennoch vielen Angriffen stand oder schrecken bereits vor einem Versuch ab. Während große Sicherheitsfirmen hierfür teure Software- und Hardwarelösungen anbieten, gibt es derzeit kein besonders großes Angebot an Abwehrmaßnahmen, die der Prävention von manipulativen Angriffen auf die Mitarbeiter dienen.

Es liegt also auf der Hand, dass immer mehr Angriffe nicht mehr herkömmlich auf die IT-Systeme direkt gerichtet werden, sondern entsprechende Mitarbeiter als Ziel haben. Der zu konventionellen Methoden gesparte Aufwand ist größer als er erwartet wird. Im nächsten Kapitel werden einige gängige Methoden und deren Effizienz vorgestellt.

\subsection{Gängige Angrifffe}\label{gangigeangriffe}
In diesem Kapitel werden exemplarisch einige Angriffstechniken vorgestellt und es wird grob geklärt, warum diese Techniken besonders große Erfolgschancen bieten. Dabei werden einige psychologische Grundlagen vorweggenommen, welche in Kapitel 3 detailliert beschrieben werden. Zum Verständnis der folgenden Beispiele sind diese Grundlagen nicht notwendig, jedoch werden nach der Lektüre des Kapitels Psychologische Grundlagen die Vorgänge hinter diesen Beispielen noch sehr viel deutlicher erkennbar sein.

\subsubsection{Heimarbeit und Helpdesks}\label{heimarbeitundhelpdesks}
Ein beliebtes Ziel für Social Engineering Angriffe stellen Mitarbeiter dar, die von zu Hause aus arbeiten. Dabei können die Mitarbeiter direkt als Ziel genommen werden. Mitarbeiter, die zu großen Teilen von zu Hause aus arbeiten, wissen oftmals nicht über alle Kollegen Bescheid und können auf Anrufe eines Angreifers, der sich als vermeintlicher Arbeitskollege ausgibt, vertrauliche Informationen herausgeben. Dabei wird Distanz zum Unternehmen als Schwachstelle ausgenutzt.

Dieses Angriffsziel ist allerdings zweischneidiger Natur. Auch in die andere Richtung können Angriffe vollzogen werden. Dabei wird der IT-Helpdesk als Zielscheibe gewählt. Mitarbeiter eines solchen Helpdesks sind oft geschult darauf freundlich und zuvorkommend zu handeln. Da gegenüber Heimarbeitern eine besonders große Hilfsbereitschaft an den Tag gelegt wird, ist es eine beliebte Taktik sich als Heimarbeiter auszugeben und so gewünschte Informationen zu erhalten.

\subsubsection{Neue Angestellte}\label{neueangestellte}
Dieses Szenario ist zwar etwas seltener anzutreffen aber durchaus nicht zu unterschätzen, da auf diese Weise bereits Zugriffsrechte auf bestimmte Bereiche des Unternehmens oder der Datenbasis gewährt wird. Auch in solchen Situationen wird die Hilfsbereitschaft von anderen Mitarbeitern ausgenutzt, denn gerade neue Mitarbeiter benötigen am Anfang viel Hilfe. Außerdem bringt man neuen Mitarbeitern mehr Nachsehen entgegen.

In einer solchen Position ist es für einen Angreifer ein leichtes an eine Vielzahl wichtiger Informationen zu kommen oder zusätzliche Schwachstellen ausfindig zu machen.

Wenn neue Mitarbeiter nach kurzer Zeit ohne plausiblen Grund Kündigen, ist dies oft ein Indiz dafür, dass es sich um einen Angriff gehandelt haben kann. Um solche Situationen zu verhindern, ist es angebracht für neue Mitarbeiter ausreichende Background-Checks vorzunehmen.

\subsubsection{Angriff auf Mitwisser}\label{angriffaufmitwisser}
Manchmal stellt es den Angreifer vor große Probleme das Ziel direkt zu attakieren, da evtl. keine brauchbaren Sicherheitslücken ausfindig gemacht werden können oder das Risiko eines direkten Angriffes zu hoch wäre. Hier bietet es sich für den Angreifer an sich ein kooperierendes Unternehmen als Ziel zu wählen, welches größere Sicherheitslücken aufweist. Die Informationen die Dritte über das Zielunternehmen haben, können entscheidend für den tatsächlichen Angriff bieten. Auf ein solches drittes Unternehmen können dann Methoden wie bereits beschrieben angewendet werden.

\subsubsection{Kombination sozialer und technischer Angriffe}\label{kombination}
 Oft ist mit rein manipulativen Mitteln das gewünschte Ziel nicht erreichbar. Dabei nutzen Angreifer häufig eine Mischform von Social Engineering Techniken und herkömmlichen Angriffstechniken. Dafür kann ein Angreifer beispielsweise einen mit Malware infizierten USB-Stick in einer Büroetage liegen lassen. Die Neugier einiger Mitarbeiter wird dabei größer sein als die Vernunft und es ist keine seltenheit, dass ein solcher USB-Stick den Weg in die Buchse eines Mitarbeiter-PCs findet.

EIne sehr bekannte technsiche Methode, die sich Social Engineering Techniken bedient, ist das Senden von Phishing-Mails. Dabei werden an eine große Menge von Zielen E-Mails versendet, die sich verschiedene menschlichen Schwächen zu Nutze machen wie z.B. Leichtgläubigkeit, Kurzsichtigkeit oder Gier. Diese Technik findet allerdings weniger Anwendung bei Angriffen auf Unternehmen. Der Ausbeute eines solchen Angriffs ergibt sich nicht aus einem einzigen Angriff sondern vielmehr aus der Menge. Tatsächlich ist es eine Minderheit im Promillbereich, die auf solche Phishing-Mails reagiert. Sendet der Angreifer 1000000 Mails aus, bei denen pro Aktion \EUR{1000} erbeutet werden können, genügt eine Erfolgsrate von 0,01\% um 100 erfolgreiche Angriffe verbuchen zu können, was einem Gewinn von \EUR{100000} entspricht.