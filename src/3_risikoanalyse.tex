\section{Risikoanalyse und -management}\label{sec:risikoanalyse}
In den vorigen Kapiteln ist das Prinzip von Social Engineering erläutert worden und somit klar
geworden wie leicht solche Angriffe von statten gehen können und welche Gefahren im Vergleich zu
herkömmlichen Angriffsmethoden bestehen.
Dennoch wird das von Social Engineering herrührende Risiko oft unterschätzt.
Um diesen Gefahren zielgerichtet entgegenzuwirken ist es wichtig ein grundlegendes, allgemeines
Verständnis von Risikofaktoren, deren Bewertungen und der Handhabung mit ihnen zu bekommen.
Dieses Kapitel vermittelt zunächst wobei es sich bei dem Begriff Risiko im Allgemeinen handelt.
Es wird außerdem geschildert wie man mit bereits erkannten Risiken umgehen kann (Kapitel 3.1.1) und
aus welchen Gründen Risiken meist kein angemessenes Risikomanagement seine Anwendung findet.
Abschließend werden diese allgemeinen Erkenntnisse auf den Bereich der speziellen Risikofaktoren durch
Social Engineering Angriffe übertragen (Kapitel 3.2).

\subsection{Allgemeines}\label{sec:allgemeines}
Risiken im Allgemeinen und ihre Handhabung treten in nahezu allen Situationen des täglichen Lebens
auf.
Dabei kann es sich um solche Situationen trivialer Natur handeln wie der Wahl eines Getränks bis hin
zur Entscheidung für oder gegen die Investition einer Aktie.
Auch wenn beide Situationen sehr unterschiedlich anmuten, haben sie doch sehr viel gemeinsam.
Sowohl die Wahl eines Getränks als auch die Investition in eine Aktie bilden ein gewisses Risiko.
Offensichtlich bieten die beiden Aktionen unterschiedliche große Risiken, weswegen es notwendig ist
solche Risiken korrekt einzustufen und zu bewerten.
Ein wichtiges Merkmal für ein Risiko ist, dass ein jedes Risiko mit einer Entscheidung verknüpft ist,
wobei es sich dabei auch um die Entscheidung nichts zu tun handeln kann.

In Bezug auf Informationssicherheit kann eine Gefahr bedeuten, dass geschäftskritische Daten für
nicht-autorisierte Nutzer zugänglich gemacht werden.
Trifft man in einem solchen Fall die Entscheidung nichts zu tun, ist das damit verbundene Risiko
entsprechend hoch.

Die Auseinandersetzung mit Entscheidungen, den damit Verbunden Risiken und den Möglichkeiten des
zweckmäßigen Umgangs ist vor allem wertvoll für die positive Beeinflussung zukünftiger Situationen.
Die Analyse von Risiken sollte nicht dazu verwendet werden Ereignisse zu erklären welche der
Vergangenheit angehören.

In Bezug auf den Umgang mit Risiken gibt es verschieden charakterisierte Herangehensweisen, welche
freilich nicht in der Reinform auftreten, aber generell oft wiederzuerkennen sind.

Die fatalistische Herangehensweise zeichnet sich besonders durch die damit verbundene Passivität aus.
Dabei wird davon ausgegangen, dass zukünftige Ereignisse weder abwendbar noch voraussehbar sind.
Auffällig ist, dass das Handeln ausschließlich aus Reaktionen besteht.

Ein anderes Extrem ist der fanatische Ansatz. Hierbei wird von einer konkreten Zukunft ausgegangen,
welche man in diesem Zusammenhang als Vision bezeichnen kann.
Abweichungen von dieser Vision in Form anderer Möglichkeiten werden gänzlich ignoriert und lösen auch
keine notwendigen Reaktionen aus.

Dass diese beiden Handlungsweisen nicht zielführend sind, ist offensichtlich.
Allerdings ist auch eine rein wissenschaftliche Beurteilung von Risikosituationen nicht von nutzen.
Ein rein wissenschaftlicher Ansatz ist zwar unvoreingenommen, zeichnet sich aber durch die
Notwendigkeit einer stichhaltigen Beweislage aus.
Da diese für in der Zukunft liegende Ereignisse nicht vorhanden ist, würde der rein wissenschaftliche
Ansatz nie zu einer endgültigen Entscheidung führen.

Der pragmatische Ansatz geht von der Annahme aus, dass die Zukunft zwar ungewiss ist, aber nicht
gänzlich unvorhersehbar ist.
Dabei sollen die Chancen positiv verändert werden, während eine mangelnde Beweislage akzeptiert wird.

Dan Borge S. 1-8, 21-39

\subsection{Risikomanagement}\label{sub:risikomanagement}
Der Umgang mit Risiken setzt sich aus mehreren Schritten zusammen.
Der erste aber auch schwierigste Schritt ist die Identifizierung von Risiken.
Die Schwierigkeit besteht im Grunde darin, dass nicht klar ist, wonach überhaupt gesucht wird.
Das liegt an der bislang noch spärlichen Kategorisierung von Risiken.
In der Informationssicherheit sind jedoch viele Risiken bereits bekannt, obgleich sich die Methoden
schnell weiterentwickeln.
Ein Risiko stellt auch die Bedrohung durch Social Engineering Angriffe dar.

Gängige Werke zum Thema Risikomanagement empfehlen beim Umgang mit Risiken verschiedene Taktiken.
Dabei wird beispielsweise die Vermeidung, das Verkaufen und Verteilen der Risiken vorgeschlagen.
Im Falle von gespeicherten Informationen stellen diese Lösungsansätze allerdings keine brauchbaren
Alternativen dar.
Zwei Möglichkeiten für den Umgang mit Risiken solcher Art sind zum einen die Versicherung für den
Fall, dass ein Schadensfall eintritt, zum anderen die gezielte Absicherung.
Da mit dem Verlust solcher Daten an Dritte nicht nur finanzieller Schaden sondern auch ein immenser
Imageschaden einhergeht, ist auch die Herangehensweise mit einer Versicherung nicht zweckgemäß.
In aller Regel bildet nur ein gezielter Schutz vor Angriffen eine angemessene Art des
Risikomanagements.

// to do: Vorgehensweise für Risk-Management in IT-Sec recherchieren

Dan Borge S. 45-58

\subsection{Ursachen für schlechtes Risikomanagement}\label{sec:ursachen}
Oft stehen dem korrekten Umgang mit Risiken allerdings einige Hürden im Weg. Zum einen sind es sicherlich Schätzungen und Vereinfachungen, die eine Fehlerquelle bilden. Viel gefährlicher sind jedoch solche Fehlerquellen menschlicher Natur. Diese Hindernisse zur rationalen Entscheidungsfindung gründen auf grundlegenden Eigenschaften der menschlichen Psyche (welche wiederum die Grundlage für die Techniken des Social Engineering bildet).

Menschen neigen grundsätzlich dazu die eigenen Möglichkeiten zu überschätzen.
Möglichkeiten, denen eine sehr geringe Wahrscheinlichkeit zugeordnet ist, werden oft ausgeblendet und
als unmöglich bezeichnet.
Die Tendenz bei Menschen geht allerdings dahin, dass dies stärker bei Risiken geschieht als bei der
Wahrnehmung von Möglichkeiten.
Die Ursache hierfür ist das Phänomen des menschlichen Optimismus.
Negative Resultate werden vielmals unterschätzt und unverhältnismäßig hohe Risiken werden für Chancen
mit geringer Aussicht auf Erfolg aufgenommen.

Des Weiteren ist oft eine fälschliche Betrachtung vergangener Ereignisse zu beobachten.
Im Detail bedeutet das, dass eingetretenen Ereignissen der Vergangenheit im Nachhinein nachgesagt
wird, sie seien absehbar gewesen, auch wenn sie vor ihrem Eintreten als unwahrscheinlich eingeordnet
worden sind.
Mit dieser rückblickenden Betrachtung werden aus optimistischen Herangehensweise resultierende
Fehlentscheidungen gerechtfertigt und beschönigt.

Vielen Menschen fällt es außerdem schwer an die Beliebigkeit und die Zufälligkeit von Ereignissen zu
glauben.
Tatsächlich neigen sie dazu nach Mustern zu suchen und Ereignisse in eine Ordnung zu bringen.
So werden beispielsweise aus einer Entwicklung über eine relativ kurze Zeit Rückschlüsse für die Zukunft geschlossen.
Ähnlich dem Suchen von Mustern geht die Kurzsichtigkeit bei der Beurteilung von Ereignissen einher.
Dies ist unter anderem bei Fußballtrainern zu beobachten.
Schon eine kurze Serie an Niederlagen, verleitet Fans und Vereinsvorstand die Ursache beim Trainer zu
suchen.
Viele Trainer werden schon nach kurzen Zeiträumen wieder entlassen, ohne dass eine solide
Informationsbasis zur rationalen Entscheidungsfindung existiert.

Einen Fußballtrainer nach einer Serie von Niederlagen zu entlassen, ist zwar eine harte
Entscheidungsfindung, aber aus risikoanalytischen Gesichtspunkten immer noch sinnvoller als die
Trägheit des menschlichen Verhaltens.
Die Ursache hierfür liegt darin, dass Menschen eine falsche Aktion negativer bewerten als das
Unterlassen einer Aktion, auch wenn die Passivität das schlechtere Ergebnis liefert.
Gerade in Bezug auf die Informationssicherheit ist es undenkbar keine Gegenmaßnahmen zu treffen oder
im Falle eines Informationsverlustes in Tatenlosigkeit zu verharren.

Eine Bekannte Ursache für eine Fehleinschätzung von Risiken sind ironischerweise
Sicherheitsmechanismen wie der Sicherheitsgurt im Auto oder ein verbesserter Algorithmus zur
Berechnung des Risikos für Finanzanlagen.
Eben diese risikominimierenden Algorithmen sorgen dafür, dass sich der Anwender in Sicherheit fühlt
und damit in Summe ein größeres Risiko eingeht, als er ohne diese Absicherung eingegangen wäre.

Der letzte signifikante Punkt ist die Selbstzufriedenheit des Menschen. Darunter ist zu verstehen,
dass dem Menschen bekannte Risiken weniger gefährlich erscheinen als solche, die über seine Gewohnheit
hinausgehen.
Gerade in Bezug auf Social Engineering ist dieser Punkt erwähnenswert, denn es ist dem Menschen in der
Tat eine Gewohnheit mit anderen Menschen zu Interagieren.
So fällt es Menschen schwer das Risiko zu erkennen, welches von einem Gespräch ausgehen kann.
Grundsätzliche Skepsis ist allerdings ebenso kontraproduktiv und eine es muss abgeschätzt werden
inwiefern Sicherheitsvorkehrungen bei internen und externen Gesprächen eines Unternehmens sinnvoll
sind.
Das Finden dieser Balance wird im Kapitel 10 Gegenmaßnahmen genauer erörtert.


// to do: weitere Recherche Tversky, Kahnemann

Dan Borge S. 61-72

\subsection{Social Engineering als Risikofaktor}\label{sec:risikofaktor}
Bis zu diesem Kapitel wurde das Phänomen Social Engineering und die grundlegende Idee dahinter
beschrieben sowie der Umgang mit Risikofaktoren und den Hindernissen der menschlichen Psyche.

Social Engineering bildet unbestreitbar einen beachtenswerten Risikofaktor.
Dabei gibt es eine Vielzahl an möglichen Szenarien, die beobachtet und analysiert werden können.
Leider ist es zu oft der Fall, dass erfolgreiche Angriffe die auf Social Engineering basieren in den
seltensteten Fällen als solche identifiziert werden.
Gerade weil Angriffe dieser Art oft in Kombination mit technischen Methoden angewandt werden, wird ein
rein technischer Angriff dokumentiert.
Aus diesem Grund und eines allgemein fahrlässigen Risikomanagements wegen wird in vielen Unternehmen
die Gefahr von Social Engineering nicht angemessen ernst genommen.