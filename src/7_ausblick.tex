\section{Diskussion und Ausblick}\label{diskussion-und-ausblick}

Im Verlauf der vorliegenden Arbeit wurde zunächst an das Thema Social Engineering herangeführt und einige grundlegende Angriffsmuster beschrieben.
Die mit solchen Angriffen erreichte Effizienz ist erschreckend und ist ein klares Zeichen für Unternehmen diese Bedrohung ernst zu nehmen.
Vor allem die diffizile forensische Aufklärung solcher Angriffe macht es schwer genaue Zahlen zu erfassen, denn oftmals stehen hinter scheinbar rein technischen Angriffen die Methoden eines Social Engineers am Anfang der Kausalkette.
Zwar ist in sehr großen Konzernen die Gefahr bereits erkannt und korrekt eingeschätzt, jedoch ist es vor allem in kleinen und mittelständischen Unternehmen ein noch nicht ernst genug genommenes Thema.
Hacking Angriffe zielen zwar meist auf militärische Einrichtungen, Banken oder ähnliche Institutionen ab, jedoch liegt das Risiko für andere Branchen bei weitem nicht bei null.
Wie das Kapitel über \nameref{sec:risikoanalyse} gezeigt hat, werden viele Risiken wie Social Engineering oft nicht ernst genug genommen und zu oft darauf gehofft, dass ein negatives Ergebnis ausbleibt.
Gewissenhafte Datenschutzbeauftragte in Firmen sollten demnach überlegen das Unternehmen und seine Mitarbeiter gegen Angriffe dieser Art zu schützen.
Problematisch können an dieser Stelle selbstverständlich die aufkommenden Kosten für eine solide Implementierung von Abwehrmaßnahmen sein.
Es kann jedoch aktiv ein Verständnis der Vertraulichkeit von Daten gelebt werden, was schon mit wenigen Mitteln stark verbessert werden kann.

Die Umfrage bestätigt in einigen Punkten die Dringlichkeit einer intensiveren Aufklärung.
Zwar gaben nur wenige Probanden an einen Mitarbeiter mit defekten/fehlenden ID-Chip einzulassen, jedoch wurde auf Bitten anderer Szenarien weitaus freizügiger reagiert.
Zwar gibt es leichte Schwankungen zwischen Berufsgruppen welche in der EDV angesiedelt sind und anderen Berufsgruppen, diese beziehen sich jedoch größtenteils auf Situationen in denen es konkret um Aktivitäten an IT-Geräten geht.
Wenn es sich um Zugangskontrollen oder ähnliches handelt gleichen sich die Ergebnisse an.
Sicherlich bietet das Grundprinzip solcher Umfragen noch weiteres Potenzial zur genaueren Untersuchung.
Einzelne Zielgruppen können genauer analysiert werden, um somit genauere Vorstellungen von den Schwachstellen aufzudecken.

