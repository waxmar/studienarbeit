\section{Psychologische Grundlagen}\label{sec:psychologische_grundlagen}

\subsection{Kommunikation}

\subsection{Vertrauen}
% Warum vertrauen wir Mitmenschen?
% Warum ist Vertrauen wichtig?
% Hang zum Kooperieren.
% Vertrauen vor Manipulation?

\subsection{Instrumente der Manipulation}\label{sec:instrumente_der_manipulation}
Bei Social Engineering geht es wie eingangs in \Kapitel{sec:social_engineering} bereits erwähnt vordergründig
nicht darum, den Zugriff auf sensible Daten durch gezielte technische Angriffe gewährt zu bekommen,
für die man selbst keine Berechtigung verfügt.
Vielmehr geht es darum andere Personen, welche für den Zugriff auf die gewünschten Informationen, dahingehend
zu beeinflussen, Aktionen durchzuführen, die dem Angreifer entweder diese Daten zuspielen oder ihn sogar selbst
dafür zu autorisieren.
Für solche Angriffe bieten sich einige bewährte Methoden an, wie sie unter anderem auch in der Werbebranche
Verwendung finden.
Dabei wird auf verschiedene Mechanismen und Automatismen des menschlichen Verhaltens abgezielt, wie sie ab \Kapitel{sec:reziprozität} erläutert werden sollen.
Es wird dafür in jedem der folgenden Abschnitte eine kurze Erklärung des Prinzips gegeben sowie die Art und
Intensität der Wirkung aufgezeigt und eine Angriffsmöglichkeit im Social Engineering erläutert.
Diese Mechanismen liegen zunächst alle dem Prinzip der Automatismen zugrunde, welches in folgenden Kapitel beschrieben wird.

\subsubsection{Automatismen}\label{sec:automatismen}
In der Psychologie versteht man unter einem Automatismus eine vom Bewusstsein nicht kontrolliert ablaufende Tätigkeit.
Bei Lebewesen handelt es sich dabei im Speziellen um fest im Unterbewusstsein verankerte Verhaltensmuster.
Entwickelt haben sich diese Mechanismen im Laufe der Evolution und sind somit ein fester Bestandteil der Psyche aller Lebewesen.
Diese Automatismen bestehen im Wesentlichen aus zwei Teilen. Zunächst ist ein bestimmtes Ereignis nötig. Dabei kann sich um einfache Reize aller Art handeln (visuell, akustisch, haptisch, etc.) oder auch um ein
komplexes Zusammenspiel mehrerer solcher Reize.
Dieses Ereignis löst unwillkürlich eine fest zugeordnete Reaktion aus, welche den zweiten Teil des Automatismus darstellt.
Bei allen Lebewesen gibt es diese Automatismen. \name{Robert Cialdini} führt dafür in seinem Buch \"Die Psychologie des Überzeugens\" das Beispiel einer Truthenne an. Diese reagiert auf ein bestimmtes Geräusch, welches ihre Küken von sich geben (akustischer Reiz), damit, dass sie ihre Küken füttert.
Auf den ersten Blick ist dieser Automatismus auch wirksam und hilfreich, denn er erlaubt das herausfiltern irrelevanter Informationen.
Zwar erspart sich die Henne eine aufwändige Betrachtung der gesamten Informationen, jedoch ist sie dadurch bereits anfällig für eine Art Social Engineering, da Teilinformationen, die auf einen Täuschungsversuch hindeuten können übersehen werden.
Legt man der Henne einen Lautsprecher vor, der eben diese Geräusche von sich gibt, wird sie das gleiche Verhalten zeigen, wie wenn es sich um ein echtes Küken handelt.

Diese Verhaltensweisen sind bei Tieren zwar besonders stark ausgeprägt, machen allerdings bei uns Menschen nicht gänzlich halt.
Denn auch Menschen benötigten in der Vergangenheit solche Automatismen um zu überleben und benötigen diese auch heute noch.
Der Unterschied zu den Tieren besteht allerdings darin, dass es Menschen leichter fällt diese Automatismen abzulegen.
Nichtsdestoweniger bieten Automatismen auch beim Menschen Potenzial für Social Engineering Angriffe.
In Anbetracht dessen ist es in der Praxis vielmehr das Zusammenspiel vieler Faktoren, die bestimmte Verhaltensweisen auslösen. Konkret sind damit Verhaltensmuster gemeint, welche erst durch die Bildung von Gemeinschaften entstanden sind. Dabei handelt es sich um erlernte Automatismen, die uns den Umgang mit Mitmenschen erleichtern. Aber auch diese Automatismen können ausgenutzt werden, um ein gewünschtes Verhalten auszulösen. \cite{cialdini}


\subsubsection{Reziprozität}\label{sec:reziprozität}
Reziprozität leitet sich vom lateinischen Wort \fremdwort{reziprocus} ab, was mit \translation{wechselseitig} zu übersetzen ist.
Die Wechselseitigkeit tritt dergestalt auf, dass durch Leitungen, Gefälligkeiten o.ä. beim Empfänger derselben das Gefühl entsteht, sich dafür revanchieren zu müssen.
Die Nützlichkeit dieses Mechanismus ist nicht bestreitbar.
So ist es für uns selbstverständlich Dienstleistungen finanziell zu entlohnen, was für eine funktionierende Gesellschaft, wie wir sie kennen, unabdingbar ist.
Das gleiche gilt für Gefälligkeiten, welche nicht finanziell vergütet werden. Ein Beispiel hierfür ist es, das Anliegen eines Arbeitskollegen bevorzugt zu behandeln. Dadurch dass der anderen Person ein Gefallen erwiesen wird, entsteht bei ihr das Gefühl sich revanchieren zu müssen.
Dieses Phänomen tritt in vielen verschiedenen Variationen auf, welche meist auf einer zuvorkommenden Handlungsweise, einer Art Präsent oder auch einem eigenen Zugeständnis beruhen. Die Intensität der Auswirkung, die das Prinzip der Reziprozität mit sich bringt, hängt in jedem Fall vom Wert der erwiesenen Gefälligkeit bzw. des erbrachten Geschenks ab.

Die bis hierhin beschriebenen Fakten, erwecken allerdings längst keine Skepsis, da sie uns Menschen in den meisten Fällen als logische Konsequenz erscheinen. Allerdings kann Reziprozität auch gegen die Interessen eines Menschen genutzt werden. Eine Vielzahl von Verkäufern machen sich die Macht der Reziprozität zu Nutze. 
Dabei wird bspw. mit einem kleinen Geschenk in Form eines \ugspr{Kennenlernpakets} o.ä. der potenzielle Kunde dahingehend beeinflusst, einem späteren Kaufangebot zuzusagen, da er \ugspr{in der Schuld} des Verkäufers steht. \cite{cialdini}

Dieses Szenario ist nur ein Beispiel von vielen. Grundsätzlich ist jedoch zu erkennen, dass das Ausnutzen der Reziprozität immer das Ziel hat, bestimmte Handlungen beim Opfer auszulösen. Im Kontext des Social Engineering, kann dies die Preisgabe kritischer Daten sein. Wie bei einem im vorigen Kapitel beschriebenen Automatismus, liegt die Gefahr der Reziprozität darin, dass Menschen im täglichen gesellschaftlichen Leben darauf angewiesen sind und demnach von ihrer Korrektheit überzeugt sind.

% noch ein konkretes Beispiel anführen!

\subsubsection{Konsistenz}
%Was ist Konsistenz
Der Begriff Konsistenz bezieht sich im Kontext der Soziologie auf den logischen Zusammenhang von Worten, Meinungen und Taten einer Person. Menschen haben gesellschaftsbedingt das Bedürfnis in dem was sie tun, sagen und glauben konsistent zu sein. Der Grund dafür liegt darin, dass in unserer Gesellschaft ein hohes Ansehen genießt, wer verlässlich und grundsatztreu handelt. Menschen, die regelmäßig ihre Meinung ändern, gelten gemeinhin als unberechenbar und nicht vertrauenswürdig.
Persönliche Konsistenz stellt zudem auch in der täglichen Entscheidungsfindung ein verlässliches Werkzeug dar. So lassen sich mit überschaubarem zeitlichen Aufwand gute Entscheidungen treffen.

Die Zeitersparnis ergibt sich daraus, dass nicht mehr alle relevanten Informationen überprüft werden und genau an dieser Stelle liegt die Gefahr einer bedingungslos konsistenten oder auch konsequenten Handlungsweise. Diese können auch dazu genutzt werden, bestimmte Aktionen einer Person zu erzwingen. \cite{cialdini}

Ein bekannter Überzeugungstrick ist es, die Zielperson zu einem Statement zu bringen. Durch diese Aussage nimmt sie eine Position ein. Das Prinzip der Konsistenz führt nun dazu, dass diese Person in folgenden Handlungen durch dieses Statement beeinflusst wird. Davon hängt letztlich auch ab, ob die Zielperson der Bitte des Angreifers nachkommt. Dabei ist das folgende Szenario vorstellbar:

Der Angreifer nähert sich einem Rezeptionisten in einem Krankenhaus, um Informationen (Zimmer, Zustandt, ...) über einen Patienten zu erhalten.
Das Gespräch könnte wie folgt ablaufen:

\actor{Angreifer: } \says{Guten Tag. Sie scheinen sehr gut darin zu sein, Leuten zu helfen.}
\actor{Rezeptionist: }\says{Ja, das ist mein Job.}
\actor{Angreifer: }\says{Wären Sie dann so freundlich und sagen mir in welchem Zimmer sich Patient XY aufählt?}
\actor{Rezeptionist: }\says{Natürlich. Patient XY liegt in Zimmer 4711.}

So oder so ähnlich kann sich ein solches Gespräch abspielen. Natürlich hängt es auch stark von den schauspielerischen Fähigkeiten des Angreifers ab, ob eine solche Aktion glückt.
Die Essenz dieses Angriffs ist und bleibt jedoch das Konsistenz-Prinzip. Der Rezeptionist macht hierbei eine Aussage über seine Fähigkeit Menschen eine gewünschte Auskunft zu geben. Das Nicht-Preisgeben der Information würde demnach gegen das Konsistenz-Bewusstsein der Zielperson verstoßen. Wie stark dieses Bewusstsein ist, hängt wie im vorigen Kapitel zur Reziprozität beschrieben ebenfalls von der vorangegangenen Handlung ab. In diesem Fall handelt es sich dabei um die Intensität des gemachten Statements.

Faktoren, die die Auswirkungen eines solchen Statements verstärken, sind Aktivität des Opfers (formuliert es das Statement selbst oder bestätigt es nur?), Öffentlichkeit (hören andere Leute zu?) und auch die Ungezwungenheit (hat das Opfer das Gefühl zu einem Statement gedrängt?). \cite{cialdini}

Anders als im bereits aufgeführten Beispiel, kann auch eine genaue Analyse der Zielperson Angriffspunkte offenlegen. %Hierfür ein Beipsiel

\subsubsection{Sympathie}
Die in den vorangegangenen Kapiteln beschriebenen Mechanismen lassen sich auch mit anderen \ugspr{Techniken} kombinieren. Es ist allgemein bekannt, dass es besonders solchen Menschen leicht fällt uns zu Manipulieren, welche uns sympathisch sind.
Sympathie wirkt wie ein sozialer Katalysator bei der Überzeugung anderer Menschen. 
An dieser Stelle soll zunächst erläutert werden, wie Sympathie entsteht und von welchen Faktoren sie abhängt.

Das ausschlaggebendste Kriterium welches beeinflusst, ob wir eine Person sympathisch finden, ist tatsächlich die körperliche Attraktivität. Besonders bei gut aussehenden Menschen ist der sogenannte \fremdwort{Halo-Effekt} zu beobachten (dt. Heiligenschein-Effekt). Menschen schließen von der äußerlichen Schönheit einer Person direkt auf andere (davon unabhängige) Eigenschaften wie z.B. Kompetenz, Freundlichkeit
oder Begabung.
Neben körperlicher Attraktivität ist auch die Ähnlichkeit zur Zielperson selbst ausschlaggebend. Das können sowohl äußerliche sowie charakterliche Merkmale und auch Ansichten sein.

Sympathie kann zudem durch die wiederholte Kontaktaufnahme mit der Zielperson aufgebaut werden. Dabei ist es umso förderlicher, wenn mit dem Kontakt eine erfolgreiche Kooperation einhergeht (welcher Art auch immer). Verstärken lässt sich der Effekt auch durch das Verteilen von Komplimenten. Denn durch das Erhalten eines Kompliments tritt zusätzlich der Effekt der Reziprozität in Kraft. Bei einem Kompliment handelt es um eine Art Geschenk, welches es zu erwidern gilt. So äußert die Zielperson bspw., dass sie eine bestimmte Eigenschaft der Person gegenüber schätzt. Somit ist außerdem ein Statement geäußert (\speech{Mein Gegenüber ist nett}), was das Konsistenz-Bewusstsein bedient.

Die Sympathie ist demnach bestens dafür geeignet mit anderen manipulativen Werkzeugen eingesetzt zu werden, da sie die Willfährigkeit anderer Menschen verstärkt. Die Sympathie ist gerade deshalb gefährlich, weil sie dem Menschen im Alltag ein guter Ratgeber sind, welchen Personen man trauen kann bzw. welche man besser meidet. \cite{cialdini}

\subsubsection{Autorität und Befehle}\label{sec:autorität-und-befehle}
Eine gegensätzlich und dennoch überraschend ähnliche Methode Menschen zu überzeugen ist die Macht der Autorität. In unserer Gesellschaft besteht ein starker Druck, was das befolgen von Anweisungen durch eine Autorität angeht. Damit sind zunächst einmal nur echte Autoritäten gemeint. Gefährlich wird es sobald Meinungen oder Befehle einer Autorität nur noch hingenommen und nicht mehr hinterfragt werden. Denn hinter solchen Befehlen kann sich ein Fehler verbergen oder noch schlimmer eine falsche Autorität in Form eines Social Engineers.

Es ist aus diesem Grund wichtig zu wissen, wann eine bestimmte Person für eine Autorität bzw. einen Experten gehalten wird. Meist wird nicht die eigentliche Autorität wahrgenommen sondern nur Symbole, die auf eine solche hindeuten. Bei solchen Symbolen handelt es sich im Speziellen um Kleidung (Arztkittel, Uniform, usw.), auch um Titel (Graf, Prof., Dr. med., etc.) oder andere äußerliche Merkmale wie die Körpergröße.

Die Autoritätshörigkeit, wie wir sie kennen, ist deshalb so oft vorzufinden, da Menschen unserer Gesellschaft darauf gedrillt werden Anweisungen von Autoritäten zu folgen, da diese über mehr Wissen, Macht oder Erfahrung verfügen. Zusätzlich werden Autoritäten auch zur Vereinfachung der eigenen Entscheidungsfindung herangezogen. \cite{cialdini}
